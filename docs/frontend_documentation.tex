\documentclass[12pt,a4paper]{report}
\usepackage[utf8]{inputenc}
\usepackage[T1]{fontenc}
\usepackage[french]{babel}
\usepackage{graphicx}
\usepackage{geometry}
\usepackage{xcolor}
\usepackage{tikz}
\usepackage{float}
\usepackage{hyperref}
\usepackage{listings}
\usepackage{titlesec}
\usepackage{fancyhdr}
\usepackage{booktabs}
\usepackage{array}
\usepackage{colortbl}
\usepackage{enumitem}
\usepackage{caption}
\usepackage{longtable}
\usepackage{tabularx}
\usepackage{tcolorbox}
\usepackage{pifont}
\usepackage{setspace}

\usetikzlibrary{shapes,arrows,positioning,calc,fit,backgrounds}

% Configuration
\geometry{hmargin=2.5cm,vmargin=2.5cm}
\definecolor{flutterblue}{RGB}{2, 119, 189}
\definecolor{dartblue}{RGB}{0, 180, 216}
\definecolor{clinalertgreen}{RGB}{0, 200, 83}
\definecolor{clinorange}{RGB}{255, 145, 0}
\definecolor{clinred}{RGB}{255, 82, 82}
\definecolor{clingray}{RGB}{108, 117, 125}
\definecolor{lightgray}{RGB}{248, 249, 250}
\definecolor{darkblue}{RGB}{25, 55, 109}
\definecolor{codebg}{RGB}{248, 249, 250}

\hypersetup{
    colorlinks=true,
    linkcolor=flutterblue,
    urlcolor=dartblue,
    pdftitle={ClinAlert - Documentation Frontend},
    pdfauthor={Équipe ClinAlert},
}

% En-têtes et pieds de page
\pagestyle{fancy}
\fancyhf{}
\fancyhead[L]{\small\leftmark}
\fancyhead[R]{\small\thepage}
\fancyfoot[C]{\small ClinAlert - Documentation Frontend Flutter}
\renewcommand{\headrulewidth}{0.4pt}
\renewcommand{\footrulewidth}{0.4pt}

% Format des chapitres
\titleformat{\chapter}[display]
  {\normalfont\huge\bfseries\color{flutterblue}}
  {\chaptertitlename\ \thechapter}{20pt}{\Huge}

% Style code Dart
\lstdefinestyle{dart}{
    language=Java,
    basicstyle=\small\ttfamily,
    keywordstyle=\color{flutterblue}\bfseries,
    stringstyle=\color{clinalertgreen},
    commentstyle=\color{clingray}\itshape,
    backgroundcolor=\color{codebg},
    frame=single,
    breaklines=true,
    numbers=left,
    numberstyle=\tiny\color{clingray},
    morekeywords={final,var,const,async,await,Widget,BuildContext,State,StatefulWidget,StatelessWidget,void,String,int,double,bool,List,Map,Future,dynamic,required,override}
}

% Boîtes colorées
\tcbuselibrary{skins,breakable}
\newtcolorbox{infobox}[1][]{colback=blue!5,colframe=flutterblue,fonttitle=\bfseries,title=#1,breakable}
\newtcolorbox{warningbox}[1][]{colback=orange!5,colframe=clinorange,fonttitle=\bfseries,title=#1,breakable}
\newtcolorbox{successbox}[1][]{colback=green!5,colframe=clinalertgreen,fonttitle=\bfseries,title=#1,breakable}

\begin{document}

%=============================================================================
% PAGE DE GARDE
%=============================================================================
\begin{titlepage}
    \begin{tikzpicture}[remember picture, overlay]
        % Logo Flutter
        \node[anchor=north west] at ([xshift=2cm, yshift=-2cm]current page.north west) {
            \begin{tikzpicture}
                \fill[flutterblue!80] (0,0) -- (1.5,0.75) -- (0,1.5) -- cycle;
                \fill[dartblue!80] (0,0) -- (1.5,0.75) -- (1.5,-0.5) -- (0.5,-0.5) -- cycle;
            \end{tikzpicture}
        };
        
        \node[anchor=north east, align=right] at ([xshift=-2cm, yshift=-2cm]current page.north east) {
            {\color{flutterblue}\fontfamily{phv}\selectfont\bfseries\large Documentation Technique}\\[3pt]
            {\color{clingray}\fontfamily{phv}\selectfont Frontend Flutter Mobile}
        };
        
        \draw[flutterblue, line width=1.5pt] ([xshift=2cm, yshift=-5cm]current page.north west) -- ([xshift=-2cm, yshift=-5cm]current page.north east);
        
        \node[align=center] at ([yshift=3cm]current page.center) {
            {\color{flutterblue}\fontsize{50}{55}\selectfont\bfseries ClinAlert}\\[0.8cm]
            {\color{clingray}\Large Application Mobile de Suivi Médical}
        };
        
        \node at ([yshift=0.5cm]current page.center) {\textcolor{dartblue}{\rule{10cm}{2.5pt}}};
        
        \node[align=center] at ([yshift=-2cm]current page.center) {
            {\color{darkblue}\fontsize{20}{24}\selectfont\bfseries Rapport Technique Frontend}\\[0.5cm]
            {\color{clingray}\fontsize{14}{18}\selectfont Flutter, Dart \& Architecture Provider}
        };
        
        \node at ([yshift=-5cm]current page.center) {\textcolor{dartblue}{\rule{10cm}{2.5pt}}};
        
        \node[anchor=south, align=center] at ([yshift=6cm]current page.south) {
            \begin{tabular}{rl}
                {\color{flutterblue}\textbf{Framework}} & Flutter 3.x \\[0.2cm]
                {\color{flutterblue}\textbf{Langage}} & Dart 3.x \\[0.2cm]
                {\color{flutterblue}\textbf{State Management}} & Provider \\[0.2cm]
                {\color{flutterblue}\textbf{Navigation}} & Named Routes \\[0.2cm]
                {\color{flutterblue}\textbf{Localisation}} & FR, EN, AR \\
            \end{tabular}
        };
        
        \draw[flutterblue, line width=1.5pt] ([xshift=2cm, yshift=3.5cm]current page.south west) -- ([xshift=-2cm, yshift=3.5cm]current page.south east);
        
        \node[anchor=south] at ([yshift=2cm]current page.south) {
            {\color{clingray}\fontfamily{phv}\selectfont\large\bfseries \today}
        };
    \end{tikzpicture}
\end{titlepage}

\tableofcontents
\newpage

%=============================================================================
% CHAPITRE 1: INTRODUCTION
%=============================================================================
\chapter{Introduction Générale}

\section*{Introduction du chapitre}
Ce chapitre présente le frontend de l'application ClinAlert développé avec Flutter, un framework cross-platform permettant de créer des applications mobiles performantes.

\section{Présentation de l'Application}

\textbf{ClinAlert Mobile} est une application Flutter conçue pour le suivi médical en temps réel. Elle permet aux utilisateurs (patients, médecins, infirmiers, administrateurs) d'interagir avec le système de santé de manière intuitive.

\begin{infobox}[Fonctionnalités principales]
\begin{itemize}
    \item \textbf{Authentification sécurisée} : Login/Register avec JWT
    \item \textbf{Tableau de bord adaptatif} : Interface différente selon le rôle
    \item \textbf{Connexion SmartWatch} : Bluetooth BLE pour collecte de données
    \item \textbf{Visualisation des données} : Graphiques interactifs de santé
    \item \textbf{Système d'alertes} : Notifications en temps réel
    \item \textbf{Multilingue} : Support Français, Anglais, Arabe (RTL)
\end{itemize}
\end{infobox}

\section{Stack Technologique}

\begin{table}[H]
\centering
\caption{Technologies utilisées dans le frontend}
\begin{tabular}{llp{6cm}}
\toprule
\textbf{Catégorie} & \textbf{Technologie} & \textbf{Description} \\
\midrule
Framework & Flutter 3.x & Framework UI cross-platform de Google \\
Langage & Dart 3.x & Langage orienté objet et typé \\
State Management & Provider & Gestion d'état réactive et simple \\
HTTP Client & http & Appels REST API vers le backend \\
Stockage local & SharedPreferences & Persistance du token JWT \\
Bluetooth & flutter\_blue\_plus & Communication avec SmartWatch \\
Graphiques & fl\_chart & Visualisation des données de santé \\
Localisation & flutter\_localizations & Support multilingue (FR/EN/AR) \\
\bottomrule
\end{tabular}
\end{table}

\section{Structure du Projet}

\begin{successbox}[Organisation des dossiers lib/]
\begin{itemize}
    \item \texttt{screens/} : 27 écrans de l'application
    \item \texttt{models/} : 15 modèles de données
    \item \texttt{services/} : 7 services (API, Auth, BLE, etc.)
    \item \texttt{widgets/} : 18 composants réutilisables
    \item \texttt{providers/} : 3 providers de gestion d'état
    \item \texttt{l10n/} : Fichiers de traduction (FR, EN, AR)
    \item \texttt{themes/} : Configuration du thème visuel
    \item \texttt{utils/} : Utilitaires et helpers
\end{itemize}
\end{successbox}

\section*{Conclusion du chapitre}
L'application Flutter ClinAlert offre une expérience utilisateur moderne et réactive, avec une architecture modulaire facilitant la maintenance et l'évolution.

%=============================================================================
% CHAPITRE 2: ARCHITECTURE
%=============================================================================
\chapter{Architecture de l'Application}

\section*{Introduction du chapitre}
Ce chapitre détaille l'architecture logicielle adoptée pour le frontend, basée sur le pattern Provider pour la gestion d'état.

\section{Architecture Provider}

\begin{figure}[H]
\centering
\begin{tikzpicture}[
    layer/.style={rectangle, draw=flutterblue, line width=2pt, fill=flutterblue!10, 
                  minimum width=12cm, minimum height=1.2cm, rounded corners=5pt, font=\bfseries},
    arrow/.style={->, line width=3pt, color=dartblue}
]
    \node[layer] (ui) at (0,6) {UI Layer - 27 Screens (Widgets)};
    \node[layer] (provider) at (0,4.5) {State Layer - 3 Providers (AuthProvider, etc.)};
    \node[layer] (service) at (0,3) {Service Layer - 7 Services (ApiService, etc.)};
    \node[layer] (model) at (0,1.5) {Model Layer - 15 Models (Patient, HealthData, etc.)};
    \node[layer, fill=clinalertgreen!10, draw=clinalertgreen] (api) at (0,0) {Backend API - Spring Boot REST};
    
    \draw[arrow] (ui) -- (provider);
    \draw[arrow] (provider) -- (service);
    \draw[arrow] (service) -- (model);
    \draw[arrow] (service) -- (api);
\end{tikzpicture}
\caption{Architecture en couches de l'application Flutter}
\end{figure}

\section{Flux de Données}

\begin{enumerate}
    \item L'utilisateur interagit avec un \textbf{Widget} (écran)
    \item Le Widget appelle une méthode du \textbf{Provider}
    \item Le Provider utilise un \textbf{Service} pour les opérations
    \item Le Service effectue des appels HTTP vers l'\textbf{API REST}
    \item Les données sont mappées vers les \textbf{Models}
    \item Le Provider notifie les listeners avec \texttt{notifyListeners()}
    \item L'UI se reconstruit automatiquement
\end{enumerate}

\section{Gestion de l'État avec Provider}

\begin{lstlisting}[style=dart, caption=Exemple AuthProvider]
class AuthProvider extends ChangeNotifier {
  User? _currentUser;
  String? _token;
  bool _isLoading = false;
  
  User? get currentUser => _currentUser;
  bool get isAuthenticated => _token != null;
  
  Future<bool> login(String email, String password) async {
    _isLoading = true;
    notifyListeners();
    
    try {
      final response = await _authService.login(email, password);
      _token = response['token'];
      _currentUser = User.fromJson(response);
      
      await _storageService.saveToken(_token!);
      notifyListeners();
      return true;
    } catch (e) {
      return false;
    } finally {
      _isLoading = false;
      notifyListeners();
    }
  }
}
\end{lstlisting}

\section{Navigation par Rôle}

L'application redirige vers différents dashboards selon le rôle de l'utilisateur :

\begin{table}[H]
\centering
\caption{Écrans de dashboard par rôle}
\begin{tabular}{|l|l|l|}
\hline
\textbf{Rôle} & \textbf{Écran} & \textbf{Fonctionnalités} \\
\hline
ADMIN & SettingsScreen & Gestion utilisateurs, cliniques \\
\hline
DOCTOR & DoctorDashboardScreen & Liste patients, alertes, stats \\
\hline
NURSE & NurseDashboardScreen & Saisie mesures, suivi patients \\
\hline
PATIENT & PatientDashboardScreen & Données perso, SmartWatch \\
\hline
\end{tabular}
\end{table}

\section*{Conclusion du chapitre}
L'architecture Provider permet une gestion d'état simple et efficace, avec une séparation claire entre l'UI, la logique métier et les données.

%=============================================================================
% CHAPITRE 3: ÉCRANS DE L'APPLICATION
%=============================================================================
\chapter{Écrans de l'Application}

\section*{Introduction du chapitre}
Ce chapitre présente les 27 écrans de l'application ClinAlert, organisés par fonctionnalité.

\section{Écrans d'Authentification}

\begin{table}[H]
\centering
\caption{Écrans d'authentification}
\begin{tabularx}{\textwidth}{|l|X|}
\hline
\textbf{Écran} & \textbf{Description} \\
\hline
WelcomeScreen & Page d'accueil avec options Login/Register \\
\hline
LoginScreen & Formulaire de connexion avec email/password \\
\hline
SignupScreen & Inscription avec sélection du rôle \\
\hline
ForgotPasswordScreen & Réinitialisation du mot de passe \\
\hline
CreateProfileScreen & Création du profil après inscription \\
\hline
\end{tabularx}
\end{table}

\section{Tableaux de Bord}

\begin{table}[H]
\centering
\caption{Dashboards par rôle utilisateur}
\begin{tabularx}{\textwidth}{|l|X|}
\hline
\textbf{Écran} & \textbf{Description} \\
\hline
DoctorDashboardScreen & Vue médecin avec liste patients, alertes récentes, statistiques globales \\
\hline
NurseDashboardScreen & Vue infirmier avec patients assignés, formulaires de saisie \\
\hline
PatientDashboardScreen & Vue patient avec données personnelles, connexion SmartWatch, graphiques \\
\hline
\end{tabularx}
\end{table}

\section{Gestion des Patients}

\begin{table}[H]
\centering
\caption{Écrans de gestion des patients}
\begin{tabularx}{\textwidth}{|l|X|}
\hline
\textbf{Écran} & \textbf{Description} \\
\hline
PatientsScreen & Liste des patients avec recherche et filtres \\
\hline
PatientDetailScreen & Détails complets d'un patient avec onglets \\
\hline
PatientHistoryScreen & Historique des mesures et consultations \\
\hline
AddEditPatientScreen & Formulaire d'ajout/modification patient \\
\hline
HealthDataScreen & Graphiques interactifs des données de santé \\
\hline
\end{tabularx}
\end{table}

\section{SmartWatch et Données}

\begin{table}[H]
\centering
\caption{Écrans SmartWatch}
\begin{tabularx}{\textwidth}{|l|X|}
\hline
\textbf{Écran} & \textbf{Description} \\
\hline
SmartWatchConnectionScreen & Appairage Bluetooth avec détection \\
\hline
BleScanScreen & Scan des appareils BLE disponibles \\
\hline
MeasurementScreen & Affichage des mesures en temps réel \\
\hline
RecordVitalScreen & Saisie manuelle des signes vitaux \\
\hline
\end{tabularx}
\end{table}

\section{Administration}

\begin{table}[H]
\centering
\caption{Écrans d'administration}
\begin{tabularx}{\textwidth}{|l|X|}
\hline
\textbf{Écran} & \textbf{Description} \\
\hline
SettingsScreen & Paramètres utilisateur, thème, langue \\
\hline
UsersManagementScreen & CRUD des comptes utilisateurs (Admin) \\
\hline
ClinicsScreen & Gestion des cliniques et établissements \\
\hline
DoctorsScreen & Gestion des médecins \\
\hline
AlertsScreen & Liste et gestion des alertes \\
\hline
\end{tabularx}
\end{table}

\section*{Conclusion du chapitre}
Les 27 écrans couvrent l'ensemble des fonctionnalités métier, offrant une expérience utilisateur complète et intuitive.

%=============================================================================
% CHAPITRE 4: MODÈLES DE DONNÉES
%=============================================================================
\chapter{Modèles de Données}

\section*{Introduction du chapitre}
Ce chapitre présente les 15 modèles Dart utilisés pour représenter les données échangées avec l'API backend.

\section{Vue d'Ensemble des Modèles}

\begin{table}[H]
\centering
\caption{Liste complète des modèles Flutter}
\begin{tabular}{llc}
\toprule
\textbf{Modèle} & \textbf{Description} & \textbf{Taille} \\
\midrule
User & Compte utilisateur authentifié & 2.7 KB \\
Patient & Patient suivi dans le système & 3.4 KB \\
Doctor & Médecin avec spécialité & 2.2 KB \\
Clinic & Établissement médical & 1.6 KB \\
HealthData & Données de santé (12 métriques) & 3.9 KB \\
DailyHealthSummary & Résumé quotidien calculé & 5.9 KB \\
Alert & Alerte médicale générée & 3.3 KB \\
SmartWatchDevice & Appareil connecté & 2.7 KB \\
VitalSign & Signe vital individuel & 3.7 KB \\
Measurement & Mesure ponctuelle & 1.0 KB \\
Message & Message chat & 3.6 KB \\
\bottomrule
\end{tabular}
\end{table}

\section{Modèle HealthData}

\begin{lstlisting}[style=dart, caption=Modèle HealthData complet]
class HealthData {
  final String? id;
  final String patientId;
  final String? deviceId;
  final int? heartRate;
  final double? spO2;
  final int? steps;
  final int? sleepMinutes;
  final int? bloodPressureSystolic;
  final int? bloodPressureDiastolic;
  final double? temperature;
  final int? caloriesBurned;
  final double? distanceMeters;
  final DateTime timestamp;
  final String? source;
  
  factory HealthData.fromJson(Map<String, dynamic> json) {
    return HealthData(
      id: json['id'],
      patientId: json['patientId'],
      heartRate: json['heartRate'],
      spO2: json['spO2']?.toDouble(),
      steps: json['steps'],
      sleepMinutes: json['sleepMinutes'],
      timestamp: DateTime.parse(json['timestamp']),
      source: json['source'],
    );
  }
  
  Map<String, dynamic> toJson() => {
    'patientId': patientId,
    'heartRate': heartRate,
    'spO2': spO2,
    'steps': steps,
    'sleepMinutes': sleepMinutes,
    'timestamp': timestamp.toIso8601String(),
    'source': source,
  };
}
\end{lstlisting}

\section{Modèle User avec Rôles}

\begin{lstlisting}[style=dart, caption=Modèle User avec énumération des rôles]
enum UserRole { ADMIN, DOCTOR, NURSE, PATIENT }

class User {
  final String id;
  final String email;
  final UserRole role;
  final String? firstName;
  final String? lastName;
  final String? phone;
  final bool enabled;
  
  bool get isAdmin => role == UserRole.ADMIN;
  bool get isDoctor => role == UserRole.DOCTOR;
  bool get isPatient => role == UserRole.PATIENT;
  
  factory User.fromJson(Map<String, dynamic> json) {
    return User(
      id: json['id'],
      email: json['email'],
      role: UserRole.values.firstWhere(
        (r) => r.name == json['role'],
        orElse: () => UserRole.PATIENT,
      ),
      firstName: json['firstName'],
      lastName: json['lastName'],
      enabled: json['enabled'] ?? true,
    );
  }
}
\end{lstlisting}

\section*{Conclusion du chapitre}
Les modèles Dart reflètent fidèlement les entités du backend, avec sérialisation JSON complète pour les échanges API.

%=============================================================================
% CHAPITRE 5: SERVICES
%=============================================================================
\chapter{Services et Communication API}

\section*{Introduction du chapitre}
Ce chapitre présente les 7 services responsables de la communication avec le backend et des fonctionnalités locales.

\section{Vue d'Ensemble des Services}

\begin{table}[H]
\centering
\caption{Services de l'application}
\begin{tabularx}{\textwidth}{|l|X|c|}
\hline
\textbf{Service} & \textbf{Responsabilité} & \textbf{Taille} \\
\hline
ApiService & Appels REST vers le backend (50+ endpoints) & 20.7 KB \\
\hline
AuthService & Authentification, login, register, logout & 7.3 KB \\
\hline
BleService & Communication Bluetooth avec SmartWatch & 3.7 KB \\
\hline
StorageService & Persistance locale (token, préférences) & 1.7 KB \\
\hline
MessageService & Gestion des messages et chat & 12.7 KB \\
\hline
NotificationService & Notifications push locales & 0.6 KB \\
\hline
ExportService & Export de données (PDF, CSV) & 0.7 KB \\
\hline
\end{tabularx}
\end{table}

\section{ApiService - Communication REST}

\begin{lstlisting}[style=dart, caption=Extrait ApiService avec authentification]
class ApiService {
  static const String baseUrl = 'http://10.0.2.2:8080/api';
  
  Future<Map<String, String>> _getHeaders() async {
    final token = await StorageService.getToken();
    return {
      'Content-Type': 'application/json',
      'Authorization': token != null ? 'Bearer $token' : '',
    };
  }
  
  // === PATIENTS ===
  Future<List<Patient>> getPatients() async {
    final response = await http.get(
      Uri.parse('$baseUrl/patients'),
      headers: await _getHeaders(),
    );
    
    if (response.statusCode == 200) {
      final List<dynamic> data = json.decode(response.body);
      return data.map((e) => Patient.fromJson(e)).toList();
    }
    throw Exception('Failed to load patients');
  }
  
  // === HEALTH DATA ===
  Future<List<HealthData>> getPatientHealthData(String patientId) async {
    final response = await http.get(
      Uri.parse('$baseUrl/smartwatch/health-data/$patientId'),
      headers: await _getHeaders(),
    );
    
    if (response.statusCode == 200) {
      final List<dynamic> data = json.decode(response.body);
      return data.map((e) => HealthData.fromJson(e)).toList();
    }
    return [];
  }
  
  Future<void> submitHealthData(List<HealthData> dataList) async {
    final response = await http.post(
      Uri.parse('$baseUrl/smartwatch/health-data'),
      headers: await _getHeaders(),
      body: json.encode(dataList.map((e) => e.toJson()).toList()),
    );
    
    if (response.statusCode != 201) {
      throw Exception('Failed to submit health data');
    }
  }
}
\end{lstlisting}

\section{BleService - Connexion SmartWatch}

\begin{lstlisting}[style=dart, caption=Service Bluetooth BLE]
class BleService {
  final FlutterBluePlus _flutterBlue = FlutterBluePlus();
  
  Stream<List<ScanResult>> scanForDevices() {
    FlutterBluePlus.startScan(timeout: Duration(seconds: 10));
    return FlutterBluePlus.scanResults;
  }
  
  Future<void> connectToDevice(BluetoothDevice device) async {
    await device.connect(autoConnect: false);
    List<BluetoothService> services = await device.discoverServices();
    // Recherche du service de donnees de sante
    for (var service in services) {
      for (var char in service.characteristics) {
        if (char.properties.notify) {
          await char.setNotifyValue(true);
          char.value.listen((data) {
            _processHealthData(data);
          });
        }
      }
    }
  }
  
  void _processHealthData(List<int> rawData) {
    // Parsing des donnees du SmartWatch
    final heartRate = rawData[0];
    final steps = (rawData[1] << 8) | rawData[2];
    // Envoyer au backend via ApiService
  }
}
\end{lstlisting}

\section*{Conclusion du chapitre}
Les services encapsulent toute la logique de communication, offrant une couche d'abstraction propre pour les providers et les widgets.

%=============================================================================
% CHAPITRE 6: INTERFACE UTILISATEUR
%=============================================================================
\chapter{Interface Utilisateur}

\section*{Introduction du chapitre}
Ce chapitre présente les aspects visuels de l'application : widgets réutilisables, thème et localisation.

\section{Widgets Réutilisables}

L'application utilise 18 widgets personnalisés pour garantir une cohérence visuelle :

\begin{table}[H]
\centering
\caption{Widgets personnalisés}
\begin{tabularx}{\textwidth}{|l|X|}
\hline
\textbf{Widget} & \textbf{Usage} \\
\hline
PatientCard & Carte affichant les infos patient avec avatar \\
\hline
HealthMetricCard & Carte métrique avec icône et valeur \\
\hline
AlertBadge & Badge coloré selon sévérité de l'alerte \\
\hline
VitalSignChart & Graphique fl\_chart pour données vitales \\
\hline
LoadingIndicator & Indicateur de chargement personnalisé \\
\hline
CustomTextField & Champ de saisie avec validation \\
\hline
GradientButton & Bouton avec dégradé de couleurs \\
\hline
StatCard & Carte statistique avec tendance \\
\hline
\end{tabularx}
\end{table}

\section{Thème et Couleurs}

\begin{lstlisting}[style=dart, caption=Configuration du thème]
class AppTheme {
  static ThemeData lightTheme = ThemeData(
    primaryColor: Color(0xFF2962FF),
    colorScheme: ColorScheme.light(
      primary: Color(0xFF2962FF),
      secondary: Color(0xFF00C853),
      error: Color(0xFFFF5252),
    ),
    scaffoldBackgroundColor: Color(0xFFF8F9FA),
    cardTheme: CardTheme(
      elevation: 4,
      shape: RoundedRectangleBorder(
        borderRadius: BorderRadius.circular(16),
      ),
    ),
    appBarTheme: AppBarTheme(
      elevation: 0,
      backgroundColor: Colors.transparent,
      foregroundColor: Color(0xFF2D3436),
    ),
  );
  
  static ThemeData darkTheme = ThemeData.dark().copyWith(
    primaryColor: Color(0xFF448AFF),
    // ...
  );
}
\end{lstlisting}

\section{Localisation Multilingue}

L'application supporte 3 langues avec support RTL pour l'arabe :

\begin{table}[H]
\centering
\caption{Langues supportées}
\begin{tabular}{|l|l|l|}
\hline
\textbf{Langue} & \textbf{Code} & \textbf{Direction} \\
\hline
Français & fr & LTR \\
\hline
Anglais & en & LTR \\
\hline
Arabe & ar & RTL \\
\hline
\end{tabular}
\end{table}

\begin{lstlisting}[style=dart, caption=Utilisation des traductions]
// Dans un widget
Text(AppLocalizations.of(context)!.welcomeMessage)
Text(AppLocalizations.of(context)!.patientList)
Text(AppLocalizations.of(context)!.alerts)
\end{lstlisting}

\section*{Conclusion du chapitre}
L'interface utilisateur respecte les principes Material Design tout en offrant une identité visuelle cohérente et une expérience multilingue.

%=============================================================================
% CHAPITRE 7: DIAGRAMMES UML
%=============================================================================
\chapter{Diagrammes UML}

\section*{Introduction du chapitre}
Ce chapitre présente les diagrammes UML illustrant l'architecture et les flux de l'application Flutter.

\section{Diagramme de Classes - Couche Modèle}

\begin{figure}[H]
\centering
\resizebox{\textwidth}{!}{
\begin{tikzpicture}[
    classbox/.style={rectangle, draw=#1, line width=2pt, fill=white, minimum width=4.5cm, inner sep=0pt, rounded corners=3pt},
    assoc/.style={->, line width=2pt, >=stealth, #1}
]

% User
\node[classbox=flutterblue] (user) at (0,8) {
    \begin{tabular}{c}
    \cellcolor{flutterblue!25} \textbf{User} \\
    \hline
    \texttt{+ id: String} \\
    \texttt{+ email: String} \\
    \texttt{+ role: UserRole} \\
    \texttt{+ firstName: String?} \\
    \hline
    \texttt{+ fromJson()} \\
    \texttt{+ toJson()} \\
    \end{tabular}
};

% Patient
\node[classbox=clinalertgreen] (patient) at (-5,4) {
    \begin{tabular}{c}
    \cellcolor{clinalertgreen!25} \textbf{Patient} \\
    \hline
    \texttt{+ id: String} \\
    \texttt{+ name: String} \\
    \texttt{+ age: int} \\
    \texttt{+ doctorId: String} \\
    \hline
    \texttt{+ fromJson()} \\
    \end{tabular}
};

% HealthData
\node[classbox=clinred] (health) at (0,4) {
    \begin{tabular}{c}
    \cellcolor{clinred!20} \textbf{HealthData} \\
    \hline
    \texttt{+ heartRate: int?} \\
    \texttt{+ spO2: double?} \\
    \texttt{+ steps: int?} \\
    \texttt{+ timestamp: DateTime} \\
    \hline
    \texttt{+ fromJson()} \\
    \texttt{+ toJson()} \\
    \end{tabular}
};

% Alert
\node[classbox=clinorange] (alert) at (5,4) {
    \begin{tabular}{c}
    \cellcolor{clinorange!20} \textbf{Alert} \\
    \hline
    \texttt{+ id: String} \\
    \texttt{+ type: String} \\
    \texttt{+ severity: String} \\
    \texttt{+ message: String} \\
    \hline
    \texttt{+ fromJson()} \\
    \end{tabular}
};

% SmartWatch
\node[classbox=dartblue] (watch) at (0,0) {
    \begin{tabular}{c}
    \cellcolor{dartblue!20} \textbf{SmartWatchDevice} \\
    \hline
    \texttt{+ deviceAddress: String} \\
    \texttt{+ deviceName: String} \\
    \texttt{+ isActive: bool} \\
    \hline
    \texttt{+ fromJson()} \\
    \end{tabular}
};

% Relations
\draw[assoc=clinalertgreen] (patient.north) -- node[left] {*} (user.south west);
\draw[assoc=clinred] (patient.east) -- node[above] {1..*} (health.west);
\draw[assoc=clinorange] (patient.south east) -- node[above] {0..*} (alert.south west);
\draw[assoc=dartblue] (patient.south) -- node[right] {0..*} (watch.north);

\end{tikzpicture}
}
\caption{Diagramme de Classes - Modèles Flutter}
\end{figure}

\section{Diagramme de Séquence - Login}

\begin{figure}[H]
\centering
\begin{tikzpicture}[scale=0.9]
    % Objects
    \node[draw, rectangle, fill=flutterblue!20, minimum width=2cm] (ui) at (0,0) {LoginScreen};
    \node[draw, rectangle, fill=dartblue!20, minimum width=2cm] (prov) at (4,0) {AuthProvider};
    \node[draw, rectangle, fill=clinalertgreen!20, minimum width=2cm] (api) at (8,0) {ApiService};
    \node[draw, rectangle, fill=clinorange!20, minimum width=2cm] (backend) at (12,0) {Backend};
    
    % Lifelines
    \draw[dashed] (ui) -- (0,-9);
    \draw[dashed] (prov) -- (4,-9);
    \draw[dashed] (api) -- (8,-9);
    \draw[dashed] (backend) -- (12,-9);
    
    % Messages
    \draw[->, thick, flutterblue] (0,-1) -- node[above, font=\small] {login(email, pwd)} (4,-1);
    \draw[->, thick, dartblue] (4,-2) -- node[above, font=\small] {authService.login()} (8,-2);
    \draw[->, thick, clinalertgreen] (8,-3) -- node[above, font=\small] {POST /api/auth/login} (12,-3);
    \draw[<--, thick, clinalertgreen] (8,-4) -- node[above, font=\small] {\{token, user\}} (12,-4);
    \draw[->, thick, clingray] (4,-5) -- node[above, font=\scriptsize] {storage.saveToken()} (4,-5.5);
    \draw[->, thick, clingray] (4,-6) -- node[above, font=\scriptsize] {notifyListeners()} (4,-6.5);
    \draw[<--, thick, dartblue] (0,-7) -- node[above, font=\small] {success} (4,-7);
    \draw[->, thick, flutterblue] (0,-8) -- node[above, font=\scriptsize] {Navigator.push(Dashboard)} (0,-8.5);
\end{tikzpicture}
\caption{Diagramme de Séquence - Processus de connexion}
\end{figure}

\section{Diagramme de Séquence - Sync SmartWatch}

\begin{figure}[H]
\centering
\begin{tikzpicture}[scale=0.85]
    % Objects
    \node[draw, rectangle, fill=flutterblue!20, minimum width=1.8cm] (screen) at (0,0) {Screen};
    \node[draw, rectangle, fill=dartblue!20, minimum width=1.8cm] (ble) at (3.5,0) {BleService};
    \node[draw, rectangle, fill=clinorange!20, minimum width=1.8cm] (watch) at (7,0) {SmartWatch};
    \node[draw, rectangle, fill=clinalertgreen!20, minimum width=1.8cm] (api) at (10.5,0) {ApiService};
    \node[draw, rectangle, fill=clingray!30, minimum width=1.8cm] (back) at (14,0) {Backend};
    
    % Lifelines
    \draw[dashed] (screen) -- (0,-10);
    \draw[dashed] (ble) -- (3.5,-10);
    \draw[dashed] (watch) -- (7,-10);
    \draw[dashed] (api) -- (10.5,-10);
    \draw[dashed] (back) -- (14,-10);
    
    % Messages
    \draw[->, thick, flutterblue] (0,-1) -- node[above, font=\scriptsize] {connect()} (3.5,-1);
    \draw[->, thick, dartblue] (3.5,-2) -- node[above, font=\scriptsize] {BLE connect} (7,-2);
    \draw[<--, thick, dartblue] (3.5,-3) -- node[above, font=\scriptsize] {connected} (7,-3);
    \draw[->, thick, dartblue] (3.5,-4) -- node[above, font=\scriptsize] {subscribe notifications} (7,-4);
    \draw[<--, thick, clinorange] (3.5,-5) -- node[above, font=\scriptsize] {health data stream} (7,-5);
    \draw[->, thick, clinalertgreen] (3.5,-6) -- node[above, font=\scriptsize] {submitHealthData()} (10.5,-6);
    \draw[->, thick, clinalertgreen] (10.5,-7) -- node[above, font=\scriptsize] {POST /health-data} (14,-7);
    \draw[<--, thick, clinalertgreen] (10.5,-8) -- node[above, font=\scriptsize] {201 Created} (14,-8);
    \draw[<--, thick, flutterblue] (0,-9) -- node[above, font=\scriptsize] {sync complete} (3.5,-9);
\end{tikzpicture}
\caption{Diagramme de Séquence - Synchronisation SmartWatch}
\end{figure}

\section*{Conclusion du chapitre}
Les diagrammes UML illustrent clairement l'architecture et les flux de données de l'application Flutter.

%=============================================================================
% CONCLUSION GÉNÉRALE
%=============================================================================
\chapter*{Conclusion Générale}
\addcontentsline{toc}{chapter}{Conclusion Générale}

\section*{Récapitulatif}

Le frontend Flutter ClinAlert offre une application mobile complète et moderne :

\begin{itemize}
    \item \ding{51} \textbf{27 écrans} : Couverture complète des fonctionnalités
    \item \ding{51} \textbf{15 modèles} : Représentation fidèle des données backend
    \item \ding{51} \textbf{7 services} : Architecture modulaire et testable
    \item \ding{51} \textbf{18 widgets} : Composants réutilisables et cohérents
    \item \ding{51} \textbf{Provider} : Gestion d'état simple et efficace
    \item \ding{51} \textbf{3 langues} : Support FR, EN, AR avec RTL
    \item \ding{51} \textbf{SmartWatch} : Intégration Bluetooth BLE complète
\end{itemize}

\section*{Améliorations Futures}

\begin{successbox}[Évolutions prévues]
\begin{enumerate}
    \item \textbf{Tests} : Tests unitaires et widgets (couverture > 80\%)
    \item \textbf{Notifications push} : Firebase Cloud Messaging
    \item \textbf{Mode hors-ligne} : Cache local avec synchronisation
    \item \textbf{Biométrie} : Authentification par empreinte/Face ID
    \item \textbf{Dark mode} : Thème sombre complet
    \item \textbf{Animations} : Transitions et micro-interactions
\end{enumerate}
\end{successbox}

\vspace{1cm}
\begin{center}
{\color{flutterblue}\rule{12cm}{2pt}}\\[0.8cm]
{\fontsize{30}{35}\selectfont\bfseries\textcolor{flutterblue}{ClinAlert Mobile}}\\[0.5cm]
{\Large\textcolor{clingray}{Application Flutter de Suivi Médical}}\\[0.3cm]
{\textcolor{dartblue}{Cross-Platform Healthcare App}}
\end{center}

\end{document}
