\documentclass[12pt,a4paper]{report}
\usepackage[utf8]{inputenc}
\usepackage[T1]{fontenc}
\usepackage[french]{babel}
\usepackage{graphicx}
\usepackage{geometry}
\usepackage{xcolor}
\usepackage{tikz}
\usepackage{float}
\usepackage{hyperref}
\usepackage{listings}
\usepackage{titlesec}
\usepackage{fancyhdr}
\usepackage{booktabs}
\usepackage{array}
\usepackage{colortbl}
\usepackage{enumitem}
\usepackage{caption}
\usepackage{longtable}
\usepackage{tabularx}
\usepackage{tcolorbox}
\usepackage{pifont}
\usepackage{setspace}

\usetikzlibrary{shapes,arrows,positioning,calc,fit,backgrounds}

% Configuration
\geometry{hmargin=2.5cm,vmargin=2.5cm}
\definecolor{clinalertblue}{RGB}{41, 98, 255}
\definecolor{clinalertgreen}{RGB}{0, 200, 83}
\definecolor{clinorange}{RGB}{255, 145, 0}
\definecolor{clinred}{RGB}{255, 82, 82}
\definecolor{clingray}{RGB}{108, 117, 125}
\definecolor{lightgray}{RGB}{248, 249, 250}
\definecolor{darkblue}{RGB}{25, 55, 109}
\definecolor{codebg}{RGB}{248, 249, 250}

\hypersetup{
    colorlinks=true,
    linkcolor=clinalertblue,
    urlcolor=clinalertgreen,
    pdftitle={ClinAlert - Documentation Backend},
    pdfauthor={Équipe ClinAlert},
}

% En-têtes et pieds de page
\pagestyle{fancy}
\fancyhf{}
\fancyhead[L]{\small\leftmark}
\fancyhead[R]{\small\thepage}
\fancyfoot[C]{\small ClinAlert - Documentation Technique Backend}
\renewcommand{\headrulewidth}{0.4pt}
\renewcommand{\footrulewidth}{0.4pt}

% Format des chapitres
\titleformat{\chapter}[display]
  {\normalfont\huge\bfseries\color{clinalertblue}}
  {\chaptertitlename\ \thechapter}{20pt}{\Huge}

% Style code
\lstdefinestyle{java}{
    language=Java,
    basicstyle=\small\ttfamily,
    keywordstyle=\color{clinalertblue}\bfseries,
    stringstyle=\color{clinalertgreen},
    commentstyle=\color{clingray}\itshape,
    backgroundcolor=\color{codebg},
    frame=single,
    breaklines=true,
    numbers=left,
    numberstyle=\tiny\color{clingray}
}

% Boîtes colorées
\tcbuselibrary{skins,breakable}
\newtcolorbox{infobox}[1][]{colback=blue!5,colframe=clinalertblue,fonttitle=\bfseries,title=#1,breakable}
\newtcolorbox{warningbox}[1][]{colback=orange!5,colframe=clinorange,fonttitle=\bfseries,title=#1,breakable}
\newtcolorbox{successbox}[1][]{colback=green!5,colframe=clinalertgreen,fonttitle=\bfseries,title=#1,breakable}

\begin{document}

%=============================================================================
% PAGE DE GARDE
%=============================================================================
\begin{titlepage}
    \begin{tikzpicture}[remember picture, overlay]
        % Logo
        \node[anchor=north west] at ([xshift=2cm, yshift=-2cm]current page.north west) {
            \begin{tikzpicture}
                \draw[clinalertblue, line width=3pt] (0,0) circle (1.2cm);
                \fill[clinalertblue!20] (0,0) circle (1cm);
                \node at (0,0) {\Huge\textcolor{clinalertblue}{$\heartsuit$}};
            \end{tikzpicture}
        };
        
        \node[anchor=north east, align=right] at ([xshift=-2cm, yshift=-2cm]current page.north east) {
            {\color{clinalertblue}\fontfamily{phv}\selectfont\bfseries\large Documentation Technique}\\[3pt]
            {\color{clingray}\fontfamily{phv}\selectfont Backend Spring Boot}
        };
        
        \draw[clinalertblue, line width=1.5pt] ([xshift=2cm, yshift=-5cm]current page.north west) -- ([xshift=-2cm, yshift=-5cm]current page.north east);
        
        \node[align=center] at ([yshift=3cm]current page.center) {
            {\color{clinalertblue}\fontsize{50}{55}\selectfont\bfseries ClinAlert}\\[0.8cm]
            {\color{clingray}\Large Système de Suivi Médical Intelligent}
        };
        
        \node at ([yshift=0.5cm]current page.center) {\textcolor{clinalertgreen}{\rule{10cm}{2.5pt}}};
        
        \node[align=center] at ([yshift=-2cm]current page.center) {
            {\color{darkblue}\fontsize{20}{24}\selectfont\bfseries Rapport Technique Backend}\\[0.5cm]
            {\color{clingray}\fontsize{14}{18}\selectfont Architecture, API REST \& Modèle de Données}
        };
        
        \node at ([yshift=-5cm]current page.center) {\textcolor{clinalertgreen}{\rule{10cm}{2.5pt}}};
        
        \node[anchor=south, align=center] at ([yshift=6cm]current page.south) {
            \begin{tabular}{rl}
                {\color{clinalertblue}\textbf{Framework}} & Spring Boot 3.2.0 \\[0.2cm]
                {\color{clinalertblue}\textbf{Langage}} & Java 17 LTS \\[0.2cm]
                {\color{clinalertblue}\textbf{Base de données}} & PostgreSQL 15 \\[0.2cm]
                {\color{clinalertblue}\textbf{Sécurité}} & JWT + Spring Security \\
            \end{tabular}
        };
        
        \draw[clinalertblue, line width=1.5pt] ([xshift=2cm, yshift=3.5cm]current page.south west) -- ([xshift=-2cm, yshift=3.5cm]current page.south east);
        
        \node[anchor=south] at ([yshift=2cm]current page.south) {
            {\color{clingray}\fontfamily{phv}\selectfont\large\bfseries \today}
        };
    \end{tikzpicture}
\end{titlepage}

\tableofcontents
\newpage

%=============================================================================
% CHAPITRE 1: INTRODUCTION
%=============================================================================
\chapter{Introduction Générale}

\section*{Introduction du chapitre}
Ce chapitre présente le contexte du projet ClinAlert, les objectifs visés et les technologies mises en œuvre pour le développement du backend.

\section{Contexte du Projet}

Dans le contexte actuel de la santé numérique, le suivi à distance des patients devient une nécessité croissante. Les avancées technologiques, notamment les appareils connectés (SmartWatch, capteurs), permettent aujourd'hui de collecter des données de santé en temps réel.

\begin{infobox}[Problématique]
Comment concevoir une plateforme capable de :
\begin{itemize}
    \item Collecter les données de santé provenant d'appareils connectés ?
    \item Détecter automatiquement les anomalies dans les signes vitaux ?
    \item Alerter les professionnels de santé en cas de risque ?
    \item Fournir un historique complet pour le suivi médical ?
\end{itemize}
\end{infobox}

\textbf{ClinAlert} répond à ces besoins en proposant une solution complète de suivi médical intelligent.

\section{Objectifs du Projet}

\begin{enumerate}
    \item \textbf{Collecte de données SmartWatch} : Réception des mesures (fréquence cardiaque, SpO2, pas, sommeil)
    \item \textbf{Gestion des patients} : CRUD complet avec association aux médecins et cliniques
    \item \textbf{Système d'alertes} : Génération automatique en cas d'anomalies détectées
    \item \textbf{Authentification sécurisée} : JWT avec gestion des rôles (Admin, Doctor, Nurse, Patient)
    \item \textbf{API REST complète} : Plus de 50 endpoints documentés
    \item \textbf{Génération de rapports} : Résumés quotidiens et bilans de santé PDF
\end{enumerate}

\section{Stack Technologique}

\begin{table}[H]
\centering
\caption{Technologies utilisées dans le backend ClinAlert}
\begin{tabular}{llp{7cm}}
\toprule
\textbf{Catégorie} & \textbf{Technologie} & \textbf{Description} \\
\midrule
Framework & Spring Boot 3.2.0 & Framework Java pour applications web modernes \\
Langage & Java 17 LTS & Version Long Term Support avec nouveautés \\
Base de données & PostgreSQL 15 & SGBD relationnel robuste et performant \\
ORM & Hibernate JPA 6.x & Mapping objet-relationnel automatisé \\
Sécurité & Spring Security & Authentification et autorisation \\
Token & JWT (jjwt 0.11.5) & Tokens stateless sécurisés \\
Build & Maven 3.x & Gestion des dépendances et build \\
PDF & iText7 7.2.5 & Génération de rapports PDF \\
\bottomrule
\end{tabular}
\end{table}

\section{Structure du Backend}

\begin{successbox}[Organisation des packages]
\begin{itemize}
    \item \texttt{model/} : 9 entités JPA (User, Patient, Doctor, Clinic, HealthData, Alert, etc.)
    \item \texttt{repository/} : 9 interfaces Spring Data JPA
    \item \texttt{service/} : 11 services de logique métier
    \item \texttt{controller/} : 8 contrôleurs REST
    \item \texttt{security/} : 5 classes de sécurité JWT
    \item \texttt{dto/} : 3 objets de transfert de données
\end{itemize}
\end{successbox}

\section*{Conclusion du chapitre}
Le projet ClinAlert utilise un stack technologique moderne et éprouvé basé sur Spring Boot et PostgreSQL pour offrir une solution de suivi médical robuste et sécurisée.

%=============================================================================
% CHAPITRE 2: ARCHITECTURE
%=============================================================================
\chapter{Architecture du Système}

\section*{Introduction du chapitre}
Ce chapitre détaille l'architecture logicielle adoptée pour le backend ClinAlert, incluant le modèle en couches et les patterns de conception utilisés.

\section{Architecture MVC-REST}

Le backend suit une architecture \textbf{MVC (Model-View-Controller)} adaptée pour une API REST. Cette architecture assure une séparation claire des responsabilités.

\begin{figure}[H]
\centering
\begin{tikzpicture}[
    layer/.style={rectangle, draw=clinalertblue, line width=2pt, fill=clinalertblue!10, 
                  minimum width=13cm, minimum height=1.3cm, rounded corners=5pt, font=\bfseries},
    arrow/.style={->, line width=3pt, color=clinalertgreen}
]
    \node[layer] (client) at (0,7.5) {Client (Application Flutter Mobile)};
    \node[layer] (controller) at (0,6) {Controller Layer - 8 Contrôleurs REST};
    \node[layer] (service) at (0,4.5) {Service Layer - 11 Services Métier};
    \node[layer] (repository) at (0,3) {Repository Layer - 9 Repositories JPA};
    \node[layer] (database) at (0,1.5) {Database Layer - PostgreSQL 15};
    
    \draw[arrow] (client) -- (controller);
    \draw[arrow] (controller) -- (service);
    \draw[arrow] (service) -- (repository);
    \draw[arrow] (repository) -- (database);
\end{tikzpicture}
\caption{Architecture en couches du backend ClinAlert}
\end{figure}

\section{Principes REST Appliqués}

L'API respecte les contraintes architecturales REST :

\begin{table}[H]
\centering
\caption{Principes REST implémentés}
\begin{tabularx}{\textwidth}{|l|X|}
\hline
\textbf{Principe} & \textbf{Implémentation dans ClinAlert} \\
\hline
Stateless & Chaque requête contient le token JWT, aucun état côté serveur \\
\hline
Uniform Interface & Endpoints standardisés : GET, POST, PUT, DELETE \\
\hline
Resource-Based & URLs basées sur les ressources : /patients, /doctors, /clinics \\
\hline
JSON & Format d'échange unique pour toutes les réponses \\
\hline
HATEOAS & Liens de navigation dans certaines réponses \\
\hline
\end{tabularx}
\end{table}

\section{Flux de Données}

\begin{enumerate}
    \item L'application Flutter envoie une requête HTTP avec le token JWT
    \item Le filtre \texttt{JwtAuthenticationFilter} valide le token
    \item Le contrôleur approprié reçoit la requête
    \item Le service exécute la logique métier
    \item Le repository accède à la base de données via JPA/Hibernate
    \item La réponse JSON est retournée au client
\end{enumerate}

\section*{Conclusion du chapitre}
L'architecture en couches garantit une maintenabilité optimale et permet une évolution facilitée du système.

%=============================================================================
% CHAPITRE 3: CAS D'UTILISATION
%=============================================================================
\chapter{Diagramme de Cas d'Utilisation}

\section*{Introduction du chapitre}
Ce chapitre présente les cas d'utilisation du système ClinAlert à travers le diagramme UML. Il identifie les acteurs et leurs interactions avec le système.

\section{Identification des Acteurs}

Le système ClinAlert identifie \textbf{quatre acteurs principaux} avec des responsabilités distinctes :

\begin{table}[H]
\centering
\caption{Description des acteurs du système ClinAlert}
\begin{tabularx}{\textwidth}{|l|l|X|}
\hline
\textbf{Acteur} & \textbf{Rôle} & \textbf{Responsabilités} \\
\hline
\textcolor{clinred}{\textbf{Administrateur}} & ADMIN & Gestion complète des utilisateurs, configuration système, supervision globale, gestion des cliniques \\
\hline
\textcolor{clinalertblue}{\textbf{Médecin}} & DOCTOR & Suivi des patients assignés, consultation des données de santé, gestion des alertes, génération de rapports \\
\hline
\textcolor{clinalertgreen}{\textbf{Infirmier}} & NURSE & Saisie des mesures manuelles, assistance au suivi quotidien des patients \\
\hline
\textcolor{clinorange}{\textbf{Patient}} & PATIENT & Consultation de ses propres données de santé, connexion de son SmartWatch, réception des alertes \\
\hline
\end{tabularx}
\end{table}

\section{Diagramme de Cas d'Utilisation UML}

\begin{figure}[H]
\centering
\resizebox{\textwidth}{!}{
\begin{tikzpicture}[
    actor/.style={circle, draw=darkblue, thick, minimum size=0.8cm, fill=white},
    actorline/.style={thick, darkblue},
    uc/.style={ellipse, draw=#1, thick, fill=#1!15, minimum width=4cm, minimum height=1.2cm, align=center, font=\small},
    systembox/.style={rectangle, draw=clinalertblue, line width=2pt, rounded corners=5pt}
]

% System boundary
\draw[systembox, fill=lightgray!30] (-1,-7) rectangle (13,9);
\node[font=\large\bfseries, color=clinalertblue] at (6,8.5) {Système ClinAlert Backend};

% ADMIN Actor
\node[actor, fill=clinred!20] (adminhead) at (-3,7) {};
\draw[actorline] (-3,6.6) -- (-3,5.8);
\draw[actorline] (-3.4,6.3) -- (-2.6,6.3);
\draw[actorline] (-3,5.8) -- (-3.3,5);
\draw[actorline] (-3,5.8) -- (-2.7,5);
\node[font=\small\bfseries, color=clinred] at (-3,4.6) {Admin};

% DOCTOR Actor
\node[actor, fill=clinalertblue!20] (dochead) at (-3,2.5) {};
\draw[actorline] (-3,2.1) -- (-3,1.3);
\draw[actorline] (-3.4,1.8) -- (-2.6,1.8);
\draw[actorline] (-3,1.3) -- (-3.3,0.5);
\draw[actorline] (-3,1.3) -- (-2.7,0.5);
\node[font=\small\bfseries, color=clinalertblue] at (-3,0.1) {Médecin};

% PATIENT Actor
\node[actor, fill=clinorange!20] (pathead) at (-3,-3) {};
\draw[actorline] (-3,-3.4) -- (-3,-4.2);
\draw[actorline] (-3.4,-3.7) -- (-2.6,-3.7);
\draw[actorline] (-3,-4.2) -- (-3.3,-5);
\draw[actorline] (-3,-4.2) -- (-2.7,-5);
\node[font=\small\bfseries, color=clinorange] at (-3,-5.4) {Patient};

% Admin Use Cases
\node[uc=clinred] (uc1) at (3,7.5) {Gérer les\\utilisateurs};
\node[uc=clinred] (uc2) at (9,7.5) {Gérer les\\cliniques};
\node[uc=clinred] (uc3) at (6,6) {Configurer\\le système};

% Doctor Use Cases
\node[uc=clinalertblue] (uc4) at (3,4) {Gérer les\\patients};
\node[uc=clinalertblue] (uc5) at (9,4) {Consulter données\\de santé};
\node[uc=clinalertblue] (uc6) at (3,2) {Gérer les\\alertes};
\node[uc=clinalertblue] (uc7) at (9,2) {Générer\\rapports PDF};

% Patient Use Cases
\node[uc=clinorange] (uc8) at (3,-1) {Consulter son\\historique};
\node[uc=clinorange] (uc9) at (9,-1) {Connecter\\SmartWatch};
\node[uc=clinorange] (uc10) at (6,-3) {Voir ses\\alertes};

% Common Use Case
\node[uc=clinalertgreen] (uc11) at (6,-5.5) {S'authentifier\\(JWT)};

% Relations
\draw[->, thick, clinred] (-2.5,7) -- (uc1.west);
\draw[->, thick, clinred] (-2.5,6.8) -- (uc2.west);
\draw[->, thick, clinred] (-2.5,6.5) -- (uc3.west);

\draw[->, thick, clinalertblue] (-2.5,2.5) -- (uc4.west);
\draw[->, thick, clinalertblue] (-2.5,2.3) -- (uc5.west);
\draw[->, thick, clinalertblue] (-2.5,2) -- (uc6.west);
\draw[->, thick, clinalertblue] (-2.5,1.7) -- (uc7.west);

\draw[->, thick, clinorange] (-2.5,-3) -- (uc8.west);
\draw[->, thick, clinorange] (-2.5,-3.3) -- (uc9.west);
\draw[->, thick, clinorange] (-2.5,-3.6) -- (uc10.west);

\draw[->, thick, clinalertgreen] (-2.5,5.5) -- (uc11.north west);
\draw[->, thick, clinalertgreen] (-2.5,0) -- (uc11.west);
\draw[->, thick, clinalertgreen] (-2.5,-4.5) -- (uc11.south west);

% Include
\draw[->, dashed, gray, thick] (uc5) -- node[right, font=\scriptsize] {<<include>>} (uc11);

\end{tikzpicture}
}
\caption{Diagramme de Cas d'Utilisation UML - Système ClinAlert}
\end{figure}

\section{Description Détaillée des Cas d'Utilisation}

\subsection{Cas d'Utilisation : Gérer les Patients}
\begin{table}[H]
\centering
\begin{tabularx}{\textwidth}{|l|X|}
\hline
\textbf{Nom} & UC-04 : Gérer les patients \\
\hline
\textbf{Acteur} & Médecin (Doctor) \\
\hline
\textbf{Précondition} & L'utilisateur est authentifié avec le rôle DOCTOR \\
\hline
\textbf{Scénario principal} & 
1. Le médecin accède à la liste de ses patients \\
& 2. Il peut ajouter un nouveau patient (POST /api/patients) \\
& 3. Il peut modifier les informations (PUT /api/patients/\{id\}) \\
& 4. Il peut changer le statut (PUT /api/patients/\{id\}/status) \\
& 5. Il peut supprimer un patient (DELETE /api/patients/\{id\}) \\
\hline
\textbf{Postcondition} & Les modifications sont persistées en base de données \\
\hline
\end{tabularx}
\end{table}

\subsection{Cas d'Utilisation : Consulter Données de Santé}
\begin{table}[H]
\centering
\begin{tabularx}{\textwidth}{|l|X|}
\hline
\textbf{Nom} & UC-05 : Consulter données de santé \\
\hline
\textbf{Acteurs} & Médecin, Patient \\
\hline
\textbf{Précondition} & Authentification JWT valide \\
\hline
\textbf{Scénario principal} & 
1. L'utilisateur sélectionne un patient (médecin) ou consulte son profil (patient) \\
& 2. Le système récupère les données via GET /api/smartwatch/health-data/\{patientId\} \\
& 3. L'historique cardiaque, SpO2, pas et sommeil sont affichés \\
& 4. Des statistiques et graphiques sont générés \\
\hline
\textbf{Extensions} & 
A1. Filtrage par plage de dates (paramètres start/end) \\
& A2. Export des données en PDF \\
\hline
\end{tabularx}
\end{table}

\section*{Conclusion du chapitre}
Le diagramme de cas d'utilisation montre clairement la séparation des responsabilités entre les quatre types d'acteurs, garantissant une gestion sécurisée des accès aux données sensibles.

%=============================================================================
% CHAPITRE 4: DIAGRAMME DE CLASSES
%=============================================================================
\chapter{Diagramme de Classes}

\section*{Introduction du chapitre}
Ce chapitre présente le modèle de données du système ClinAlert à travers le diagramme de classes UML. Le modèle comprend 9 entités JPA interconnectées.

\section{Vue d'Ensemble des Entités}

\begin{table}[H]
\centering
\caption{Liste complète des entités JPA}
\begin{tabular}{llp{6cm}}
\toprule
\textbf{Entité} & \textbf{Table SQL} & \textbf{Description} \\
\midrule
User & users & Comptes utilisateurs avec authentification JWT \\
Patient & patients & Informations des patients suivis \\
Doctor & doctors & Médecins et professionnels de santé \\
Clinic & clinics & Établissements médicaux \\
HealthData & health\_data & Données de santé collectées (12 métriques) \\
Alert & alerts & Alertes médicales générées automatiquement \\
SmartWatchDevice & smartwatch\_devices & Appareils connectés des patients \\
DailyHealthSummary & daily\_health\_summaries & Résumés quotidiens calculés \\
Measurement & measurements & Mesures historiques manuelles \\
\bottomrule
\end{tabular}
\end{table}

\section{Diagramme de Classes UML}

\begin{figure}[H]
\centering
\resizebox{\textwidth}{!}{
\begin{tikzpicture}[
    classbox/.style={rectangle, draw=#1, line width=2pt, fill=white, minimum width=5.5cm, inner sep=0pt, rounded corners=3pt},
    assoc/.style={->, line width=2.5pt, >=stealth, #1},
    relabel/.style={font=\small\bfseries, fill=white, inner sep=2pt}
]

% USER
\node[classbox=clinalertblue] (user) at (0,10) {
    \begin{tabular}{c}
    \cellcolor{clinalertblue!30} \textbf{\large User} \\[2pt]
    \hline\\[-6pt]
    \texttt{- id: String (UUID)} \\
    \texttt{- email: String} \\
    \texttt{- password: String} \\
    \texttt{- role: UserRole} \\
    \texttt{- firstName: String} \\
    \texttt{- lastName: String} \\
    \texttt{- enabled: Boolean} \\
    \hline\\[-6pt]
    \texttt{+ getAuthorities()} \\
    \end{tabular}
};

% DOCTOR
\node[classbox=clinalertblue] (doctor) at (-7,4) {
    \begin{tabular}{c}
    \cellcolor{clinalertblue!25} \textbf{\large Doctor} \\[2pt]
    \hline\\[-6pt]
    \texttt{- id: String} \\
    \texttt{- name: String} \\
    \texttt{- specialty: String} \\
    \texttt{- email: String} \\
    \texttt{- phoneNumber: String} \\
    \hline\\[-6pt]
    \texttt{+ getPatients()} \\
    \end{tabular}
};

% CLINIC
\node[classbox=clinorange] (clinic) at (7,4) {
    \begin{tabular}{c}
    \cellcolor{clinorange!25} \textbf{\large Clinic} \\[2pt]
    \hline\\[-6pt]
    \texttt{- id: String} \\
    \texttt{- name: String} \\
    \texttt{- address: String} \\
    \texttt{- phone: String} \\
    \texttt{- doctorId: String} \\
    \hline\\[-6pt]
    \texttt{+ getPatientCount()} \\
    \end{tabular}
};

% PATIENT
\node[classbox=clinalertgreen] (patient) at (0,4) {
    \begin{tabular}{c}
    \cellcolor{clinalertgreen!25} \textbf{\large Patient} \\[2pt]
    \hline\\[-6pt]
    \texttt{- id: String} \\
    \texttt{- name: String} \\
    \texttt{- age: Integer} \\
    \texttt{- gender: String} \\
    \texttt{- doctorId: String} \\
    \texttt{- clinicId: String} \\
    \texttt{- status: String} \\
    \hline\\[-6pt]
    \texttt{+ getFullInfo()} \\
    \end{tabular}
};

% HEALTHDATA
\node[classbox=clinred] (healthdata) at (-7,-3) {
    \begin{tabular}{c}
    \cellcolor{clinred!20} \textbf{\large HealthData} \\[2pt]
    \hline\\[-6pt]
    \texttt{- id: String} \\
    \texttt{- patientId: String} \\
    \texttt{- deviceId: String} \\
    \texttt{- heartRate: Integer} \\
    \texttt{- spO2: Double} \\
    \texttt{- steps: Integer} \\
    \texttt{- sleepMinutes: Integer} \\
    \texttt{- temperature: Double} \\
    \texttt{- timestamp: DateTime} \\
    \hline\\[-6pt]
    \texttt{+ isNormal()} \\
    \end{tabular}
};

% ALERT
\node[classbox=clinred] (alert) at (0,-3) {
    \begin{tabular}{c}
    \cellcolor{clinred!30} \textbf{\large Alert} \\[2pt]
    \hline\\[-6pt]
    \texttt{- id: String} \\
    \texttt{- patientId: String} \\
    \texttt{- type: String} \\
    \texttt{- severity: String} \\
    \texttt{- message: String} \\
    \texttt{- read: Boolean} \\
    \texttt{- timestamp: DateTime} \\
    \hline\\[-6pt]
    \texttt{+ markAsRead()} \\
    \end{tabular}
};

% SMARTWATCH
\node[classbox=clinalertblue] (smartwatch) at (7,-3) {
    \begin{tabular}{c}
    \cellcolor{clinalertblue!20} \textbf{\large SmartWatchDevice} \\[2pt]
    \hline\\[-6pt]
    \texttt{- id: String} \\
    \texttt{- patientId: String} \\
    \texttt{- deviceAddress: String} \\
    \texttt{- deviceName: String} \\
    \texttt{- isActive: Boolean} \\
    \texttt{- lastConnected: DateTime} \\
    \hline\\[-6pt]
    \texttt{+ ping()} \\
    \end{tabular}
};

% RELATIONS
\draw[assoc=clinalertblue] (doctor.east) -- node[relabel, above] {\textbf{1}} node[relabel, below, pos=0.8] {\textbf{*}} (patient.west);
\draw[assoc=clinorange] (clinic.west) -- node[relabel, above] {\textbf{1}} node[relabel, below, pos=0.2] {\textbf{*}} (patient.east);
\draw[assoc=clinred] (patient.south) -- ++(0,-0.5) -| node[relabel, pos=0.25, left] {\textbf{1}} node[relabel, pos=0.8, left] {\textbf{*}} (healthdata.north);
\draw[assoc=clinred] (patient.south) -- node[relabel, right, pos=0.7] {\textbf{0..*}} (alert.north);
\draw[assoc=clinalertblue] (patient.south) -- ++(0,-0.3) -| node[relabel, pos=0.75, right] {\textbf{0..*}} (smartwatch.north);
\draw[->, line width=2pt, dashed, gray] (smartwatch.west) -- node[relabel, above] {\textit{génère}} (healthdata.east);

% Legend
\node[draw=clingray, thick, fill=lightgray, rounded corners, inner sep=8pt] at (0,-7) {
    \begin{tabular}{cl}
    \textcolor{clinalertblue}{\rule{1.2cm}{3pt}} & Association \\
    \textcolor{clinred}{\rule{1.2cm}{3pt}} & Composition \\
    \textcolor{gray}{- - -} & Dépendance \\
    \end{tabular}
};

\end{tikzpicture}
}
\caption{Diagramme de Classes UML - Modèle de données ClinAlert}
\end{figure}

\section{Entité HealthData Détaillée}

L'entité \texttt{HealthData} est centrale au système et contient 12 métriques de santé :

\begin{lstlisting}[style=java, caption=Entité HealthData avec toutes les métriques]
@Entity
@Table(name = "health_data")
public class HealthData {
    @Id
    @GeneratedValue(strategy = GenerationType.UUID)
    private String id;
    
    @Column(name = "patient_id", nullable = false)
    private String patientId;
    
    @Column(name = "device_id")
    private String deviceId;
    
    // Metriques de sante
    private Integer heartRate;           // bpm (40-200)
    private Double spO2;                 // % (0-100)
    private Integer steps;               // pas quotidiens
    private Integer sleepMinutes;        // minutes de sommeil
    private Integer bloodPressureSystolic;   // mmHg
    private Integer bloodPressureDiastolic;  // mmHg
    private Double temperature;          // Celsius
    private Integer caloriesBurned;      // kcal
    private Double distanceMeters;       // metres
    
    private LocalDateTime timestamp;
    private String source;  // "smartwatch", "manual"
    
    @PrePersist
    protected void onCreate() {
        receivedAt = LocalDateTime.now();
    }
}
\end{lstlisting}

\section*{Conclusion du chapitre}
Le modèle de données est organisé autour de l'entité Patient, avec des relations vers les données de santé, les alertes et les appareils connectés. L'utilisation de UUID garantit l'unicité et facilite la distribution.

%=============================================================================
% CHAPITRE 5: DIAGRAMME DE SÉQUENCE
%=============================================================================
\chapter{Diagrammes de Séquence}

\section*{Introduction du chapitre}
Ce chapitre illustre les interactions dynamiques entre les composants du système à travers des diagrammes de séquence UML pour les scénarios principaux.

\section{Séquence : Authentification JWT}

\begin{figure}[H]
\centering
\begin{tikzpicture}[scale=0.95]
    % Objects
    \node[draw, rectangle, fill=clinorange!20, minimum width=2cm] (client) at (0,0) {Client};
    \node[draw, rectangle, fill=clinalertblue!20, minimum width=2cm] (auth) at (4,0) {AuthController};
    \node[draw, rectangle, fill=clinalertblue!20, minimum width=2cm] (svc) at (8,0) {AuthService};
    \node[draw, rectangle, fill=clinalertgreen!20, minimum width=2cm] (jwt) at (12,0) {JwtProvider};
    
    % Lifelines
    \draw[dashed] (client) -- (0,-9);
    \draw[dashed] (auth) -- (4,-9);
    \draw[dashed] (svc) -- (8,-9);
    \draw[dashed] (jwt) -- (12,-9);
    
    % Messages
    \draw[->, thick, clinalertblue] (0,-1) -- node[above, font=\small] {POST /api/auth/login} (4,-1);
    \draw[->, thick, clinalertblue] (4,-2) -- node[above, font=\small] {login(email, password)} (8,-2);
    \draw[->, thick, clingray] (8,-3) -- node[above, font=\small] {findByEmail()} node[below, font=\scriptsize] {UserRepository} (8,-3.5);
    \draw[->, thick, clingray] (8,-4) -- node[above, font=\small] {matches(password)} node[below, font=\scriptsize] {BCrypt} (8,-4.5);
    \draw[->, thick, clinalertblue] (8,-5) -- node[above, font=\small] {generateToken(user)} (12,-5);
    \draw[<--, thick, clinalertgreen] (8,-6) -- node[above, font=\small] {JWT Token} (12,-6);
    \draw[<--, thick, clinalertblue] (4,-7) -- node[above, font=\small] {LoginResponse} (8,-7);
    \draw[<--, thick, clinalertgreen] (0,-8) -- node[above, font=\small] {\{token, userId, role\}} (4,-8);
\end{tikzpicture}
\caption{Diagramme de Séquence - Authentification utilisateur}
\end{figure}

\section{Séquence : Soumission Données SmartWatch}

\begin{figure}[H]
\centering
\begin{tikzpicture}[scale=0.9]
    % Objects
    \node[draw, rectangle, fill=clinorange!20, minimum width=2cm] (watch) at (0,0) {SmartWatch};
    \node[draw, rectangle, fill=clinalertblue!20, minimum width=2cm] (ctrl) at (4,0) {Controller};
    \node[draw, rectangle, fill=clinalertblue!20, minimum width=2cm] (svc) at (8,0) {HealthService};
    \node[draw, rectangle, fill=clinred!20, minimum width=2cm] (alert) at (12,0) {AlertService};
    
    % Lifelines
    \draw[dashed] (watch) -- (0,-11);
    \draw[dashed] (ctrl) -- (4,-11);
    \draw[dashed] (svc) -- (8,-11);
    \draw[dashed] (alert) -- (12,-11);
    
    % Messages
    \draw[->, thick, clinalertblue] (0,-1) -- node[above, font=\small] {POST /health-data} (4,-1);
    \draw[->, thick, clinalertblue] (4,-2) -- node[above, font=\small] {saveHealthData(list)} (8,-2);
    \draw[->, thick, clingray] (8,-3) -- node[above, font=\small] {repository.saveAll()} (8,-3.5);
    \draw[->, thick, clingray] (8,-4) -- node[above, font=\small] {checkForAnomalies()} (8,-4.5);
    
    \node[draw, rectangle, fill=clinorange!20, font=\small] at (8,-5.5) {Anomalie détectée?};
    
    \draw[->, thick, clinred] (8,-7) -- node[above, font=\small] {createAlert()} (12,-7);
    \draw[->, thick, clingray] (12,-8) -- node[above, font=\small] {save(alert)} (12,-8.5);
    \draw[<--, thick, clinalertblue] (4,-9) -- node[above, font=\small] {success + count} (8,-9);
    \draw[<--, thick, clinalertgreen] (0,-10) -- node[above, font=\small] {201 Created} (4,-10);
\end{tikzpicture}
\caption{Diagramme de Séquence - Soumission des données de santé}
\end{figure}

\section{Séquence : Consultation Historique Patient}

\begin{figure}[H]
\centering
\begin{tikzpicture}[scale=0.9]
    % Objects
    \node[draw, rectangle, fill=clinorange!20, minimum width=2cm] (app) at (0,0) {App Flutter};
    \node[draw, rectangle, fill=clinalertblue!20, minimum width=2cm] (filter) at (3.5,0) {JwtFilter};
    \node[draw, rectangle, fill=clinalertblue!20, minimum width=2cm] (ctrl) at (7,0) {Controller};
    \node[draw, rectangle, fill=clinalertblue!20, minimum width=2cm] (svc) at (10.5,0) {Service};
    \node[draw, rectangle, fill=clinalertgreen!20, minimum width=2cm] (db) at (14,0) {PostgreSQL};
    
    % Lifelines
    \draw[dashed] (app) -- (0,-10);
    \draw[dashed] (filter) -- (3.5,-10);
    \draw[dashed] (ctrl) -- (7,-10);
    \draw[dashed] (svc) -- (10.5,-10);
    \draw[dashed] (db) -- (14,-10);
    
    % Messages
    \draw[->, thick, clinalertblue] (0,-1) -- node[above, font=\scriptsize] {GET /health-data/\{id\}} node[below, font=\scriptsize] {+ JWT Header} (3.5,-1);
    \draw[->, thick, clingray] (3.5,-2) -- node[above, font=\scriptsize] {validateToken()} (3.5,-2.5);
    \draw[->, thick, clinalertblue] (3.5,-3.5) -- node[above, font=\scriptsize] {forward request} (7,-3.5);
    \draw[->, thick, clinalertblue] (7,-4.5) -- node[above, font=\scriptsize] {getPatientHealthData()} (10.5,-4.5);
    \draw[->, thick, clinalertgreen] (10.5,-5.5) -- node[above, font=\scriptsize] {SELECT * FROM health\_data} (14,-5.5);
    \draw[<--, thick, clinalertgreen] (10.5,-6.5) -- node[above, font=\scriptsize] {List<HealthData>} (14,-6.5);
    \draw[<--, thick, clinalertblue] (7,-7.5) -- node[above, font=\scriptsize] {ResponseEntity} (10.5,-7.5);
    \draw[<--, thick, clinalertgreen] (0,-8.5) -- node[above, font=\scriptsize] {JSON Array} (7,-8.5);
\end{tikzpicture}
\caption{Diagramme de Séquence - Consultation de l'historique de santé}
\end{figure}

\section*{Conclusion du chapitre}
Les diagrammes de séquence illustrent le flux des données à travers les différentes couches de l'application, mettant en évidence le rôle central du filtre JWT et la logique de détection d'anomalies.

%=============================================================================
% CHAPITRE 6: SÉCURITÉ
%=============================================================================
\chapter{Sécurité et Authentification}

\section*{Introduction du chapitre}
La sécurité est primordiale dans une application de santé. Ce chapitre détaille l'implémentation de l'authentification JWT et la gestion des autorisations.

\section{Architecture de Sécurité}

\begin{figure}[H]
\centering
\begin{tikzpicture}[
    box/.style={rectangle, draw=clinalertblue, thick, fill=clinalertblue!10, minimum width=4cm, minimum height=1cm, rounded corners}
]
    \node[box] (req) at (0,4) {Requête HTTP + JWT};
    \node[box, fill=clinorange!10, draw=clinorange] (filter) at (0,2.5) {JwtAuthenticationFilter};
    \node[box, fill=clinalertgreen!10, draw=clinalertgreen] (provider) at (0,1) {JwtTokenProvider};
    \node[box] (ctrl) at (0,-0.5) {Controller Sécurisé};
    
    \draw[->, thick, clinalertblue] (req) -- (filter);
    \draw[->, thick, clinorange] (filter) -- node[right, font=\small] {Validation} (provider);
    \draw[->, thick, clinalertgreen] (provider) -- (ctrl);
\end{tikzpicture}
\caption{Flux d'authentification JWT}
\end{figure}

\section{Configuration Spring Security}

\begin{lstlisting}[style=java, caption=Configuration de la sécurité]
@Configuration
@EnableWebSecurity
@EnableMethodSecurity(prePostEnabled = true)
public class SecurityConfig {
    
    @Autowired
    private JwtAuthenticationFilter jwtAuthenticationFilter;
    
    @Bean
    public SecurityFilterChain filterChain(HttpSecurity http) {
        http
            .cors(cors -> cors.configurationSource(corsConfigSource))
            .csrf(csrf -> csrf.disable())
            .sessionManagement(session -> session
                .sessionCreationPolicy(SessionCreationPolicy.STATELESS))
            .authorizeHttpRequests(auth -> auth
                // Endpoints publics
                .requestMatchers("/api/auth/**").permitAll()
                .requestMatchers("/error").permitAll()
                // Endpoints par role
                .requestMatchers("/api/admin/**").hasRole("ADMIN")
                .requestMatchers("/api/doctor/**").hasAnyRole("ADMIN", "DOCTOR")
                // Autres endpoints proteges
                .anyRequest().authenticated()
            )
            .addFilterBefore(jwtAuthenticationFilter, 
                UsernamePasswordAuthenticationFilter.class);
        
        return http.build();
    }
    
    @Bean
    public PasswordEncoder passwordEncoder() {
        return new BCryptPasswordEncoder();
    }
}
\end{lstlisting}

\section{Matrice des Permissions}

\begin{table}[H]
\centering
\caption{Matrice d'accès aux endpoints par rôle}
\begin{tabular}{|l|c|c|c|c|}
\hline
\textbf{Endpoint} & \textbf{ADMIN} & \textbf{DOCTOR} & \textbf{NURSE} & \textbf{PATIENT} \\
\hline
/api/auth/* & \ding{51} & \ding{51} & \ding{51} & \ding{51} \\
\hline
/api/users/* & \ding{51} & \ding{55} & \ding{55} & \ding{55} \\
\hline
/api/clinics/* & \ding{51} & \ding{51} & \ding{55} & \ding{55} \\
\hline
/api/patients/* & \ding{51} & \ding{51} & \ding{51} & \ding{55} \\
\hline
/api/smartwatch/* & \ding{51} & \ding{51} & \ding{51} & \ding{51}* \\
\hline
/api/alerts/* & \ding{51} & \ding{51} & \ding{51} & \ding{51}* \\
\hline
\end{tabular}
\end{table}

\textit{* Le patient peut uniquement accéder à ses propres données}

\begin{warningbox}[Sécurité des secrets]
En production, les secrets suivants DOIVENT être externalisés via variables d'environnement :
\begin{itemize}
    \item \texttt{JWT\_SECRET} : Clé secrète pour signer les tokens
    \item \texttt{DATABASE\_PASSWORD} : Mot de passe PostgreSQL
    \item \texttt{HMAC\_SECRET} : Clé pour l'intégrité des données
\end{itemize}
\end{warningbox}

\section*{Conclusion du chapitre}
L'authentification JWT combinée à Spring Security garantit une sécurité robuste pour toutes les communications API, avec une gestion fine des rôles.

%=============================================================================
% CHAPITRE 7: API REST
%=============================================================================
\chapter{Documentation API REST}

\section*{Introduction du chapitre}
Ce chapitre documente tous les endpoints REST disponibles dans l'API ClinAlert, organisés par contrôleur.

\section{Endpoints Authentification}

\begin{table}[H]
\centering
\caption{API Authentification (/api/auth)}
\begin{tabularx}{\textwidth}{|l|l|X|}
\hline
\textbf{Méthode} & \textbf{Endpoint} & \textbf{Description} \\
\hline
POST & /login & Authentification et génération JWT \\
\hline
POST & /register & Inscription nouvel utilisateur \\
\hline
GET & /me & Profil utilisateur connecté \\
\hline
\end{tabularx}
\end{table}

\section{Endpoints SmartWatch (20 endpoints)}

\begin{table}[H]
\centering
\caption{API SmartWatch (/api/smartwatch)}
\begin{tabularx}{\textwidth}{|l|l|X|}
\hline
\textbf{Méthode} & \textbf{Endpoint} & \textbf{Description} \\
\hline
\multicolumn{3}{|c|}{\textbf{Gestion des appareils}} \\
\hline
POST & /devices & Enregistrer un nouvel appareil \\
\hline
GET & /devices/\{patientId\} & Liste des appareils du patient \\
\hline
GET & /devices/\{patientId\}/active & Appareils actifs uniquement \\
\hline
PUT & /devices/\{deviceId\}/deactivate & Désactiver un appareil \\
\hline
DELETE & /devices/\{deviceId\} & Supprimer un appareil \\
\hline
PUT & /devices/\{deviceId\}/ping & Mettre à jour lastConnected \\
\hline
\multicolumn{3}{|c|}{\textbf{Données de santé}} \\
\hline
POST & /health-data & Soumettre un lot de données \\
\hline
POST & /health-data/single & Soumettre une mesure unique \\
\hline
GET & /health-data/\{patientId\} & Historique complet \\
\hline
GET & /health-data/\{patientId\}/range & Historique par période \\
\hline
GET & /health-data/\{patientId\}/heart-rate & Historique cardiaque \\
\hline
GET & /health-data/\{patientId\}/spo2 & Historique SpO2 \\
\hline
GET & /health-data/\{patientId\}/steps & Historique des pas \\
\hline
GET & /health-data/\{patientId\}/sleep & Historique sommeil \\
\hline
GET & /health-data/\{patientId\}/stats & Statistiques calculées \\
\hline
\multicolumn{3}{|c|}{\textbf{Résumés quotidiens}} \\
\hline
POST & /daily-summary/\{patientId\}/generate & Générer un résumé \\
\hline
GET & /daily-summary/\{patientId\} & Résumés récents \\
\hline
GET & /daily-summary/\{patientId\}/\{date\} & Résumé d'une date \\
\hline
GET & /daily-summary/\{patientId\}/range & Résumés par période \\
\hline
\end{tabularx}
\end{table}

\section{Endpoints Patients}

\begin{table}[H]
\centering
\caption{API Patients (/api/patients)}
\begin{tabularx}{\textwidth}{|l|l|X|}
\hline
\textbf{Méthode} & \textbf{Endpoint} & \textbf{Description} \\
\hline
GET & / & Liste tous les patients \\
\hline
GET & /\{id\} & Détails d'un patient \\
\hline
GET & /doctor/\{doctorId\} & Patients d'un médecin \\
\hline
GET & /clinic/\{clinicId\} & Patients d'une clinique \\
\hline
POST & / & Créer un patient \\
\hline
PUT & /\{id\} & Modifier un patient \\
\hline
PUT & /\{id\}/status & Changer le statut \\
\hline
DELETE & /\{id\} & Supprimer un patient \\
\hline
\end{tabularx}
\end{table}

\section*{Conclusion du chapitre}
L'API REST offre plus de 50 endpoints couvrant toutes les fonctionnalités du système, avec une documentation complète pour l'intégration.

%=============================================================================
% CONCLUSION GÉNÉRALE
%=============================================================================
\chapter*{Conclusion Générale}
\addcontentsline{toc}{chapter}{Conclusion Générale}

\section*{Récapitulatif}

Le backend ClinAlert offre une architecture robuste et sécurisée pour le suivi médical intelligent :

\begin{itemize}
    \item \ding{51} \textbf{Architecture MVC-REST} : Séparation claire des responsabilités en 5 couches
    \item \ding{51} \textbf{9 entités JPA} : Modèle de données complet avec 12 métriques de santé
    \item \ding{51} \textbf{8 contrôleurs REST} : Plus de 50 endpoints documentés
    \item \ding{51} \textbf{11 services} : Logique métier encapsulée avec détection d'anomalies
    \item \ding{51} \textbf{Sécurité JWT} : Authentification stateless avec gestion des rôles
    \item \ding{51} \textbf{PostgreSQL} : Base de données relationnelle fiable et performante
\end{itemize}

\section*{Améliorations Futures}

\begin{successbox}[Évolutions prévues]
\begin{enumerate}
    \item \textbf{Tests} : Ajout de tests unitaires et d'intégration (couverture > 80\%)
    \item \textbf{Rate Limiting} : Protection contre les abus API
    \item \textbf{WebSocket} : Notifications temps réel pour les alertes
    \item \textbf{Cache Redis} : Optimisation des performances de lecture
    \item \textbf{Monitoring} : Intégration Prometheus/Grafana
    \item \textbf{CI/CD} : Pipeline automatisé avec GitHub Actions
\end{enumerate}
\end{successbox}

\vspace{1cm}
\begin{center}
{\color{clinalertblue}\rule{12cm}{2pt}}\\[0.8cm]
{\fontsize{30}{35}\selectfont\bfseries\textcolor{clinalertblue}{ClinAlert}}\\[0.5cm]
{\Large\textcolor{clingray}{Healthcare Monitoring System}}\\[0.3cm]
{\textcolor{clinalertgreen}{Suivi Médical Intelligent}}
\end{center}

\end{document}
