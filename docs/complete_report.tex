\documentclass[12pt,a4paper]{report}
\usepackage[utf8]{inputenc}
\usepackage[T1]{fontenc}
\usepackage[french]{babel}
\usepackage{graphicx}
\usepackage{geometry}
\usepackage{xcolor}
\usepackage{tikz}
\usepackage{float}
\usepackage{hyperref}
\usepackage{listings}
\usepackage{titlesec}
\usepackage{fancyhdr}
\usepackage{booktabs}
\usepackage{array}
\usepackage{colortbl}
\usepackage{enumitem}
\usepackage{caption}
\usepackage{longtable}
\usepackage{tabularx}
\usepackage{tcolorbox}
\usepackage{pifont}
\usepackage{setspace}
\usepackage{lipsum}

\usetikzlibrary{shapes,arrows,positioning,calc,fit,backgrounds}

\geometry{hmargin=2.5cm,vmargin=2.5cm}
\definecolor{clinalertblue}{RGB}{41, 98, 255}
\definecolor{clinalertgreen}{RGB}{0, 200, 83}
\definecolor{clinorange}{RGB}{255, 145, 0}
\definecolor{clinred}{RGB}{255, 82, 82}
\definecolor{clingray}{RGB}{108, 117, 125}
\definecolor{lightgray}{RGB}{248, 249, 250}
\definecolor{darkblue}{RGB}{25, 55, 109}
\definecolor{codebg}{RGB}{248, 249, 250}

\hypersetup{colorlinks=true,linkcolor=clinalertblue,urlcolor=clinalertgreen,pdftitle={ClinAlert - Rapport Complet}}

\pagestyle{fancy}
\fancyhf{}
\fancyhead[L]{\small\leftmark}
\fancyhead[R]{\small\thepage}
\fancyfoot[C]{\small ClinAlert - Rapport Technique Complet}
\renewcommand{\headrulewidth}{0.4pt}
\renewcommand{\footrulewidth}{0.4pt}

\titleformat{\chapter}[display]{\normalfont\huge\bfseries\color{clinalertblue}}{\chaptertitlename\ \thechapter}{20pt}{\Huge}

\lstdefinestyle{code}{basicstyle=\small\ttfamily,keywordstyle=\color{clinalertblue}\bfseries,stringstyle=\color{clinorange},commentstyle=\color{clingray}\itshape,backgroundcolor=\color{codebg},frame=single,breaklines=true,numbers=left,numberstyle=\tiny\color{clingray},showstringspaces=false}

\tcbuselibrary{skins,breakable}
\newtcolorbox{infobox}[1][]{colback=blue!5,colframe=clinalertblue,fonttitle=\bfseries,title=#1,breakable}
\newtcolorbox{warningbox}[1][]{colback=orange!5,colframe=clinorange,fonttitle=\bfseries,title=#1,breakable}
\newtcolorbox{successbox}[1][]{colback=green!5,colframe=clinalertgreen,fonttitle=\bfseries,title=#1,breakable}

\begin{document}

% PAGE DE GARDE
\begin{titlepage}
\begin{tikzpicture}[remember picture, overlay]
\node[anchor=north west] at ([xshift=2cm, yshift=-2cm]current page.north west) {
\begin{tikzpicture}\draw[clinalertblue, line width=3pt] (0,0) circle (1.2cm);\fill[clinalertblue!20] (0,0) circle (1cm);\node at (0,0) {\Huge\textcolor{clinalertblue}{$\heartsuit$}};\end{tikzpicture}};
\node[anchor=north east, align=right] at ([xshift=-2cm, yshift=-2cm]current page.north east) {{\color{clinalertblue}\bfseries\large Documentation Technique Complète}\\[3pt]{\color{clingray} Version 1.0}};
\draw[clinalertblue, line width=1.5pt] ([xshift=2cm, yshift=-5cm]current page.north west) -- ([xshift=-2cm, yshift=-5cm]current page.north east);
\node[align=center] at ([yshift=2cm]current page.center) {{\color{clinalertblue}\fontsize{50}{55}\selectfont\bfseries ClinAlert}\\[0.8cm]{\color{clingray}\Large Système de Suivi Médical Intelligent}};
\node at ([yshift=-1cm]current page.center) {\textcolor{clinalertgreen}{\rule{10cm}{2.5pt}}};
\node[align=center] at ([yshift=-4cm]current page.center) {{\color{darkblue}\fontsize{18}{22}\selectfont\bfseries Rapport Technique Intégral}\\[0.5cm]{\color{clingray}\fontsize{14}{18}\selectfont Backend Spring Boot \& Frontend Flutter}};
\node at ([yshift=-7cm]current page.center) {\textcolor{clinalertgreen}{\rule{10cm}{2.5pt}}};
\node[anchor=south, align=center] at ([yshift=2cm]current page.south) {\begin{tabular}{rl}{\color{clinalertblue}\textbf{Backend}} & Spring Boot 3.2, Java 17, PostgreSQL \\{\color{clinalertblue}\textbf{Frontend}} & Flutter 3.x, Dart 3.x, Provider \\{\color{clinalertblue}\textbf{IoT}} & Bluetooth Low Energy (BLE) \\{\color{clinalertblue}\textbf{Date}} & \today \\\end{tabular}};
\end{tikzpicture}
\end{titlepage}

\tableofcontents
\newpage

% INTRODUCTION
\chapter{Introduction Générale}
\section{Contexte du Projet}
Dans le contexte actuel de la santé numérique, le suivi à distance des patients devient une nécessité croissante. Les avancées technologiques, notamment les appareils connectés (SmartWatch, capteurs), permettent aujourd'hui de collecter des données de santé en temps réel.

\begin{infobox}[Problématique]
Comment concevoir une plateforme capable de :
\begin{itemize}
\item Collecter les données de santé provenant d'appareils connectés ?
\item Détecter automatiquement les anomalies dans les signes vitaux ?
\item Alerter les professionnels de santé en cas de risque ?
\item Fournir un historique complet pour le suivi médical ?
\end{itemize}
\end{infobox}

\textbf{ClinAlert} répond à ces besoins en proposant une solution complète de suivi médical intelligent.

\section{Objectifs du Projet}
\begin{enumerate}
\item \textbf{Collecte de données SmartWatch} : Réception des mesures (fréquence cardiaque, SpO2, pas, sommeil)
\item \textbf{Gestion des patients} : CRUD complet avec association aux médecins et cliniques
\item \textbf{Système d'alertes} : Génération automatique en cas d'anomalies détectées
\item \textbf{Authentification sécurisée} : JWT avec gestion des rôles (Admin, Doctor, Nurse, Patient)
\item \textbf{API REST complète} : Plus de 50 endpoints documentés
\item \textbf{Génération de rapports} : Résumés quotidiens et bilans de santé PDF
\end{enumerate}

\section{Architecture Globale}
\begin{figure}[H]
\centering
\includegraphics[width=\textwidth]{images/architecture_diagram.png}
\caption{Architecture Globale du Système ClinAlert}
\end{figure}

Le système ClinAlert repose sur une architecture client-serveur moderne avec trois composants principaux :
\begin{itemize}
\item \textbf{Backend API} : Spring Boot exposant une API REST sécurisée
\item \textbf{Application Mobile} : Flutter cross-platform (Android/iOS)
\item \textbf{Objets Connectés} : SmartWatch via Bluetooth BLE
\end{itemize}

\section{Diagramme de Déploiement}
\begin{figure}[H]
\centering
\includegraphics[width=\textwidth]{images/deployment_diagram.png}
\caption{Diagramme de Déploiement UML}
\end{figure}

% PARTIE BACKEND
\part{Backend Spring Boot}

\chapter{Architecture Backend}
\section{Stack Technologique}
\begin{table}[H]
\centering
\caption{Technologies utilisées dans le backend ClinAlert}
\begin{tabular}{llp{7cm}}
\toprule
\textbf{Catégorie} & \textbf{Technologie} & \textbf{Description} \\
\midrule
Framework & Spring Boot 3.2.0 & Framework Java pour applications web modernes \\
Langage & Java 17 LTS & Version Long Term Support avec nouveautés \\
Base de données & PostgreSQL 15 & SGBD relationnel robuste et performant \\
ORM & Hibernate JPA 6.x & Mapping objet-relationnel automatisé \\
Sécurité & Spring Security & Authentification et autorisation \\
Token & JWT (jjwt 0.11.5) & Tokens stateless sécurisés \\
Build & Maven 3.x & Gestion des dépendances et build \\
PDF & iText7 7.2.5 & Génération de rapports PDF \\
\bottomrule
\end{tabular}
\end{table}

\section{Architecture en Couches}
Le backend suit une architecture \textbf{MVC (Model-View-Controller)} adaptée pour une API REST.

\begin{figure}[H]
\centering
\begin{tikzpicture}[
layer/.style={rectangle, draw=clinalertblue, line width=2pt, fill=clinalertblue!10, minimum width=13cm, minimum height=1.3cm, rounded corners=5pt, font=\bfseries},
arrow/.style={->, line width=3pt, color=clinalertgreen}]
\node[layer] (client) at (0,7.5) {Client (Application Flutter Mobile)};
\node[layer] (controller) at (0,6) {Controller Layer - 8 Contrôleurs REST};
\node[layer] (service) at (0,4.5) {Service Layer - 11 Services Métier};
\node[layer] (repository) at (0,3) {Repository Layer - 9 Repositories JPA};
\node[layer] (database) at (0,1.5) {Database Layer - PostgreSQL 15};
\draw[arrow] (client) -- (controller);
\draw[arrow] (controller) -- (service);
\draw[arrow] (service) -- (repository);
\draw[arrow] (repository) -- (database);
\end{tikzpicture}
\caption{Architecture en couches du backend ClinAlert}
\end{figure}

\section{Structure des Packages}
\begin{successbox}[Organisation des packages]
\begin{itemize}
\item \texttt{model/} : 9 entités JPA (User, Patient, Doctor, Clinic, HealthData, Alert, etc.)
\item \texttt{repository/} : 9 interfaces Spring Data JPA
\item \texttt{service/} : 11 services de logique métier
\item \texttt{controller/} : 8 contrôleurs REST
\item \texttt{security/} : 5 classes de sécurité JWT
\item \texttt{dto/} : 3 objets de transfert de données
\end{itemize}
\end{successbox}

\chapter{Modèle de Données}

\section{Diagramme Entité-Relation}
\begin{figure}[H]
\centering
\includegraphics[width=\textwidth]{images/entity_relationship_diagram.png}
\caption{Diagramme Entité-Relation de la base de données}
\end{figure}

\section{Diagramme de Classes UML}
\begin{figure}[H]
\centering
\includegraphics[width=\textwidth]{images/class_diagram_backend.png}
\caption{Diagramme de Classes UML - Modèle de données}
\end{figure}

\section{Description des Entités}
\begin{table}[H]
\centering
\caption{Liste complète des entités JPA}
\begin{tabular}{llp{6cm}}
\toprule
\textbf{Entité} & \textbf{Table SQL} & \textbf{Description} \\
\midrule
User & users & Comptes utilisateurs avec authentification JWT \\
Patient & patients & Informations des patients suivis \\
Doctor & doctors & Médecins et professionnels de santé \\
Clinic & clinics & Établissements médicaux \\
HealthData & health\_data & Données de santé collectées (12 métriques) \\
Alert & alerts & Alertes médicales générées \\
SmartWatchDevice & smartwatch\_devices & Appareils connectés \\
DailyHealthSummary & daily\_health\_summaries & Résumés quotidiens \\
Measurement & measurements & Mesures historiques manuelles \\
\bottomrule
\end{tabular}
\end{table}

\section{Entité HealthData}
L'entité \texttt{HealthData} est centrale au système et contient 12 métriques de santé :

\begin{lstlisting}[style=code, caption=Entité HealthData]
@Entity
@Table(name = "health_data")
public class HealthData {
    @Id
    @GeneratedValue(strategy = GenerationType.UUID)
    private String id;
    
    @Column(name = "patient_id", nullable = false)
    private String patientId;
    
    @Column(name = "device_id")
    private String deviceId;
    
    private Integer heartRate;           // bpm (40-200)
    private Double spO2;                 // % (0-100)
    private Integer steps;               // pas quotidiens
    private Integer sleepMinutes;        // minutes de sommeil
    private Integer bloodPressureSystolic;
    private Integer bloodPressureDiastolic;
    private Double temperature;          // Celsius
    private Integer caloriesBurned;
    private Double distanceMeters;
    
    private LocalDateTime timestamp;
    private String source;  // "smartwatch", "manual"
}
\end{lstlisting}

\section{Entité User}
\begin{lstlisting}[style=code, caption=Entité User avec roles]
@Entity
@Table(name = "users")
public class User implements UserDetails {
    @Id
    @GeneratedValue(strategy = GenerationType.UUID)
    private String id;
    
    @Column(unique = true, nullable = false)
    private String email;
    
    @Column(nullable = false)
    private String password;
    
    @Enumerated(EnumType.STRING)
    private UserRole role;  // ADMIN, DOCTOR, NURSE, PATIENT
    
    private String firstName;
    private String lastName;
    private String phone;
    private boolean enabled = true;
    
    @Override
    public Collection<? extends GrantedAuthority> getAuthorities() {
        return List.of(new SimpleGrantedAuthority("ROLE_" + role.name()));
    }
}
\end{lstlisting}

\chapter{Cas d'Utilisation}

\section{Identification des Acteurs}
\begin{table}[H]
\centering
\caption{Description des acteurs du système}
\begin{tabularx}{\textwidth}{|l|l|X|}
\hline
\textbf{Acteur} & \textbf{Rôle} & \textbf{Responsabilités} \\
\hline
\textcolor{clinred}{\textbf{Admin}} & ADMIN & Gestion utilisateurs, configuration système, cliniques \\
\hline
\textcolor{clinalertblue}{\textbf{Médecin}} & DOCTOR & Suivi patients, données de santé, alertes, rapports \\
\hline
\textcolor{clinalertgreen}{\textbf{Infirmier}} & NURSE & Saisie mesures manuelles, assistance suivi \\
\hline
\textcolor{clinorange}{\textbf{Patient}} & PATIENT & Consultation données perso, connexion SmartWatch \\
\hline
\end{tabularx}
\end{table}

\section{Diagramme de Cas d'Utilisation}
\begin{figure}[H]
\centering
\includegraphics[width=\textwidth]{images/use_case_diagram.png}
\caption{Diagramme de Cas d'Utilisation UML}
\end{figure}

\section{Cas d'Utilisation Détaillés}

\subsection{UC-01 : Gérer les Patients}
\begin{table}[H]
\centering
\begin{tabularx}{\textwidth}{|l|X|}
\hline
\textbf{Nom} & UC-01 : Gérer les patients \\
\hline
\textbf{Acteur} & Médecin (Doctor) \\
\hline
\textbf{Précondition} & Authentifié avec rôle DOCTOR \\
\hline
\textbf{Scénario} & 1. Accéder à la liste des patients \\
& 2. Ajouter un nouveau patient (POST /api/patients) \\
& 3. Modifier les informations (PUT /api/patients/\{id\}) \\
& 4. Supprimer un patient (DELETE /api/patients/\{id\}) \\
\hline
\textbf{Postcondition} & Modifications persistées en base \\
\hline
\end{tabularx}
\end{table}

\subsection{UC-02 : Consulter Données de Santé}
\begin{table}[H]
\centering
\begin{tabularx}{\textwidth}{|l|X|}
\hline
\textbf{Nom} & UC-02 : Consulter données de santé \\
\hline
\textbf{Acteurs} & Médecin, Patient \\
\hline
\textbf{Précondition} & Authentification JWT valide \\
\hline
\textbf{Scénario} & 1. Sélectionner un patient \\
& 2. GET /api/smartwatch/health-data/\{patientId\} \\
& 3. Afficher historique et graphiques \\
\hline
\end{tabularx}
\end{table}

\chapter{Diagrammes de Séquence}

\section{Séquence : Authentification JWT}
\begin{figure}[H]
\centering
\begin{tikzpicture}[scale=0.95]
\node[draw, rectangle, fill=clinorange!20, minimum width=2cm] (client) at (0,0) {Client};
\node[draw, rectangle, fill=clinalertblue!20, minimum width=2cm] (auth) at (4,0) {AuthController};
\node[draw, rectangle, fill=clinalertblue!20, minimum width=2cm] (svc) at (8,0) {AuthService};
\node[draw, rectangle, fill=clinalertgreen!20, minimum width=2cm] (jwt) at (12,0) {JwtProvider};
\draw[dashed] (client) -- (0,-9);
\draw[dashed] (auth) -- (4,-9);
\draw[dashed] (svc) -- (8,-9);
\draw[dashed] (jwt) -- (12,-9);
\draw[->, thick, clinalertblue] (0,-1) -- node[above, font=\small] {POST /api/auth/login} (4,-1);
\draw[->, thick, clinalertblue] (4,-2) -- node[above, font=\small] {login(email, password)} (8,-2);
\draw[->, thick, clingray] (8,-3) -- node[above, font=\small] {findByEmail()} (8,-3.5);
\draw[->, thick, clingray] (8,-4) -- node[above, font=\small] {matches(password)} (8,-4.5);
\draw[->, thick, clinalertblue] (8,-5) -- node[above, font=\small] {generateToken(user)} (12,-5);
\draw[<--, thick, clinalertgreen] (8,-6) -- node[above, font=\small] {JWT Token} (12,-6);
\draw[<--, thick, clinalertblue] (4,-7) -- node[above, font=\small] {LoginResponse} (8,-7);
\draw[<--, thick, clinalertgreen] (0,-8) -- node[above, font=\small] {\{token, userId, role\}} (4,-8);
\end{tikzpicture}
\caption{Diagramme de Séquence - Authentification}
\end{figure}

\section{Séquence : Soumission Données SmartWatch}
\begin{figure}[H]
\centering
\includegraphics[width=\textwidth]{images/sequence_diagram_ble.png}
\caption{Diagramme de Séquence - Soumission des données de santé}
\end{figure}

\chapter{Sécurité et Authentification}

\section{Architecture de Sécurité}
La sécurité repose sur \textbf{Spring Security} avec authentification \textbf{JWT (JSON Web Token)}.

\begin{figure}[H]
\centering
\begin{tikzpicture}[box/.style={rectangle, draw=clinalertblue, thick, fill=clinalertblue!10, minimum width=5cm, minimum height=1cm, rounded corners}]
\node[box] (req) at (0,4) {Requête HTTP + JWT};
\node[box, fill=clinorange!10, draw=clinorange] (filter) at (0,2.5) {JwtAuthenticationFilter};
\node[box, fill=clinalertgreen!10, draw=clinalertgreen] (provider) at (0,1) {JwtTokenProvider};
\node[box] (ctrl) at (0,-0.5) {Controller Sécurisé};
\draw[->, thick, clinalertblue] (req) -- (filter);
\draw[->, thick, clinorange] (filter) -- node[right, font=\small] {Validation} (provider);
\draw[->, thick, clinalertgreen] (provider) -- (ctrl);
\end{tikzpicture}
\caption{Flux d'authentification JWT}
\end{figure}

\section{Configuration Spring Security}
\begin{lstlisting}[style=code, caption=Configuration de la sécurité]
@Configuration
@EnableWebSecurity
public class SecurityConfig {
    @Bean
    public SecurityFilterChain filterChain(HttpSecurity http) {
        http
            .csrf(csrf -> csrf.disable())
            .sessionManagement(session -> session
                .sessionCreationPolicy(SessionCreationPolicy.STATELESS))
            .authorizeHttpRequests(auth -> auth
                .requestMatchers("/api/auth/**").permitAll()
                .requestMatchers("/api/admin/**").hasRole("ADMIN")
                .anyRequest().authenticated()
            )
            .addFilterBefore(jwtAuthenticationFilter, 
                UsernamePasswordAuthenticationFilter.class);
        return http.build();
    }
}
\end{lstlisting}

\section{Matrice des Permissions}
\begin{table}[H]
\centering
\caption{Matrice d'accès aux endpoints par rôle}
\begin{tabular}{|l|c|c|c|c|}
\hline
\textbf{Endpoint} & \textbf{ADMIN} & \textbf{DOCTOR} & \textbf{NURSE} & \textbf{PATIENT} \\
\hline
/api/auth/* & \ding{51} & \ding{51} & \ding{51} & \ding{51} \\
\hline
/api/users/* & \ding{51} & \ding{55} & \ding{55} & \ding{55} \\
\hline
/api/clinics/* & \ding{51} & \ding{51} & \ding{55} & \ding{55} \\
\hline
/api/patients/* & \ding{51} & \ding{51} & \ding{51} & \ding{55} \\
\hline
/api/smartwatch/* & \ding{51} & \ding{51} & \ding{51} & \ding{51}* \\
\hline
/api/alerts/* & \ding{51} & \ding{51} & \ding{51} & \ding{51}* \\
\hline
\end{tabular}
\end{table}
\textit{* Le patient peut uniquement accéder à ses propres données}

\chapter{Documentation API REST}

\section{Endpoints Authentification}
\begin{table}[H]
\centering
\caption{API Authentification (/api/auth)}
\begin{tabularx}{\textwidth}{|l|l|X|}
\hline
\textbf{Méthode} & \textbf{Endpoint} & \textbf{Description} \\
\hline
POST & /login & Authentification et génération JWT \\
\hline
POST & /register & Inscription nouvel utilisateur \\
\hline
GET & /me & Profil utilisateur connecté \\
\hline
\end{tabularx}
\end{table}

\section{Endpoints Patients}
\begin{table}[H]
\centering
\caption{API Patients (/api/patients)}
\begin{tabularx}{\textwidth}{|l|l|X|}
\hline
\textbf{Méthode} & \textbf{Endpoint} & \textbf{Description} \\
\hline
GET & / & Liste tous les patients \\
\hline
GET & /\{id\} & Détails d'un patient \\
\hline
GET & /doctor/\{doctorId\} & Patients d'un médecin \\
\hline
GET & /clinic/\{clinicId\} & Patients d'une clinique \\
\hline
POST & / & Créer un patient \\
\hline
PUT & /\{id\} & Modifier un patient \\
\hline
DELETE & /\{id\} & Supprimer un patient \\
\hline
\end{tabularx}
\end{table}

\section{Endpoints SmartWatch}
\begin{table}[H]
\centering
\caption{API SmartWatch (/api/smartwatch)}
\begin{tabularx}{\textwidth}{|l|l|X|}
\hline
\textbf{Méthode} & \textbf{Endpoint} & \textbf{Description} \\
\hline
POST & /devices & Enregistrer un nouvel appareil \\
\hline
GET & /devices/\{patientId\} & Liste des appareils du patient \\
\hline
DELETE & /devices/\{deviceId\} & Supprimer un appareil \\
\hline
POST & /health-data & Soumettre un lot de données \\
\hline
POST & /health-data/single & Soumettre une mesure unique \\
\hline
GET & /health-data/\{patientId\} & Historique complet \\
\hline
GET & /health-data/\{patientId\}/heart-rate & Historique cardiaque \\
\hline
GET & /health-data/\{patientId\}/spo2 & Historique SpO2 \\
\hline
GET & /health-data/\{patientId\}/steps & Historique pas \\
\hline
GET & /health-data/\{patientId\}/stats & Statistiques calculées \\
\hline
\end{tabularx}
\end{table}

% PARTIE FRONTEND
\part{Frontend Flutter}

\chapter{Architecture Frontend}

\section{Stack Technologique}
\begin{table}[H]
\centering
\caption{Technologies utilisées dans le frontend}
\begin{tabular}{llp{6cm}}
\toprule
\textbf{Catégorie} & \textbf{Technologie} & \textbf{Description} \\
\midrule
Framework & Flutter 3.x & Framework UI cross-platform de Google \\
Langage & Dart 3.x & Langage orienté objet et typé \\
State Management & Provider & Gestion d'état réactive \\
HTTP Client & Dio & Appels REST API vers le backend \\
Stockage local & SharedPreferences & Persistance du token JWT \\
Bluetooth & flutter\_reactive\_ble & Communication avec SmartWatch \\
Graphiques & fl\_chart & Visualisation des données de santé \\
Localisation & flutter\_localizations & Support FR/EN/AR \\
\bottomrule
\end{tabular}
\end{table}

\section{Architecture Provider}
\begin{figure}[H]
\centering
\begin{tikzpicture}[
layer/.style={rectangle, draw=clinalertblue, line width=2pt, fill=clinalertblue!10, minimum width=12cm, minimum height=1.2cm, rounded corners=5pt, font=\bfseries},
arrow/.style={->, line width=3pt, color=clinalertgreen}]
\node[layer] (ui) at (0,6) {UI Layer - 27 Screens (Widgets)};
\node[layer] (provider) at (0,4.5) {State Layer - 3 Providers (AuthProvider, etc.)};
\node[layer] (service) at (0,3) {Service Layer - 7 Services (ApiService, etc.)};
\node[layer] (model) at (0,1.5) {Model Layer - 15 Models (Patient, HealthData, etc.)};
\node[layer, fill=clinalertgreen!10, draw=clinalertgreen] (api) at (0,0) {Backend API - Spring Boot REST};
\draw[arrow] (ui) -- (provider);
\draw[arrow] (provider) -- (service);
\draw[arrow] (service) -- (model);
\draw[arrow] (service) -- (api);
\end{tikzpicture}
\caption{Architecture en couches Flutter}
\end{figure}

\section{Structure du Projet}
\begin{successbox}[Organisation des dossiers lib/]
\begin{itemize}
\item \texttt{screens/} : 27 écrans de l'application
\item \texttt{models/} : 15 modèles de données
\item \texttt{services/} : 7 services (API, Auth, BLE, etc.)
\item \texttt{widgets/} : 18 composants réutilisables
\item \texttt{providers/} : 3 providers de gestion d'état
\item \texttt{l10n/} : Fichiers de traduction (FR, EN, AR)
\item \texttt{themes/} : Configuration du thème visuel
\end{itemize}
\end{successbox}

\chapter{Écrans de l'Application}

\section{Aperçu des Interfaces}
\begin{figure}[H]
\centering
\includegraphics[width=\textwidth]{images/mobile_screens_mockup.png}
\caption{Aperçu des écrans principaux de l'application}
\end{figure}

\section{Écrans d'Authentification}
\begin{table}[H]
\centering
\caption{Écrans d'authentification}
\begin{tabularx}{\textwidth}{|l|X|}
\hline
\textbf{Écran} & \textbf{Description} \\
\hline
WelcomeScreen & Page d'accueil avec options Login/Register \\
\hline
LoginScreen & Formulaire de connexion avec email/password \\
\hline
SignupScreen & Inscription avec sélection du rôle \\
\hline
ForgotPasswordScreen & Réinitialisation du mot de passe \\
\hline
CreateProfileScreen & Création du profil après inscription \\
\hline
\end{tabularx}
\end{table}

\section{Tableaux de Bord}
\begin{table}[H]
\centering
\caption{Dashboards par rôle utilisateur}
\begin{tabularx}{\textwidth}{|l|X|}
\hline
\textbf{Écran} & \textbf{Description} \\
\hline
DoctorDashboardScreen & Vue médecin avec liste patients, alertes, statistiques \\
\hline
NurseDashboardScreen & Vue infirmier avec patients assignés, formulaires \\
\hline
PatientDashboardScreen & Vue patient avec données perso, SmartWatch, graphiques \\
\hline
\end{tabularx}
\end{table}

\section{Gestion des Patients}
\begin{table}[H]
\centering
\caption{Écrans de gestion des patients}
\begin{tabularx}{\textwidth}{|l|X|}
\hline
\textbf{Écran} & \textbf{Description} \\
\hline
PatientsScreen & Liste des patients avec recherche et filtres \\
\hline
PatientDetailScreen & Détails complets d'un patient avec onglets \\
\hline
PatientHistoryScreen & Historique des mesures et consultations \\
\hline
AddEditPatientScreen & Formulaire d'ajout/modification patient \\
\hline
HealthDataScreen & Graphiques interactifs des données de santé \\
\hline
\end{tabularx}
\end{table}

\section{SmartWatch et Données}
\begin{table}[H]
\centering
\caption{Écrans SmartWatch}
\begin{tabularx}{\textwidth}{|l|X|}
\hline
\textbf{Écran} & \textbf{Description} \\
\hline
SmartWatchConnectionScreen & Appairage Bluetooth avec détection \\
\hline
BleScanScreen & Scan des appareils BLE disponibles \\
\hline
MeasurementScreen & Affichage des mesures en temps réel \\
\hline
RecordVitalScreen & Saisie manuelle des signes vitaux \\
\hline
\end{tabularx}
\end{table}

\section{Administration}
\begin{table}[H]
\centering
\caption{Écrans d'administration}
\begin{tabularx}{\textwidth}{|l|X|}
\hline
\textbf{Écran} & \textbf{Description} \\
\hline
SettingsScreen & Paramètres utilisateur, thème, langue \\
\hline
UsersManagementScreen & CRUD des comptes utilisateurs (Admin) \\
\hline
ClinicsScreen & Gestion des cliniques et établissements \\
\hline
DoctorsScreen & Gestion des médecins \\
\hline
AlertsScreen & Liste et gestion des alertes \\
\hline
\end{tabularx}
\end{table}

\chapter{Modèles de Données Flutter}

\section{Vue d'Ensemble}
\begin{table}[H]
\centering
\caption{Liste complète des modèles Flutter}
\begin{tabular}{llc}
\toprule
\textbf{Modèle} & \textbf{Description} & \textbf{Taille} \\
\midrule
User & Compte utilisateur authentifié & 2.7 KB \\
Patient & Patient suivi dans le système & 3.4 KB \\
Doctor & Médecin avec spécialité & 2.2 KB \\
Clinic & Établissement médical & 1.6 KB \\
HealthData & Données de santé (12 métriques) & 3.9 KB \\
DailyHealthSummary & Résumé quotidien calculé & 5.9 KB \\
Alert & Alerte médicale générée & 3.3 KB \\
SmartWatchDevice & Appareil connecté & 2.7 KB \\
VitalSign & Signe vital individuel & 3.7 KB \\
Measurement & Mesure ponctuelle & 1.0 KB \\
Message & Message chat & 3.6 KB \\
\bottomrule
\end{tabular}
\end{table}

\section{Modèle HealthData}
\begin{lstlisting}[style=code, caption=Modèle HealthData Dart]
class HealthData {
  final String? id;
  final String patientId;
  final String? deviceId;
  final int? heartRate;
  final double? spO2;
  final int? steps;
  final int? sleepMinutes;
  final int? bloodPressureSystolic;
  final int? bloodPressureDiastolic;
  final double? temperature;
  final DateTime timestamp;
  final String? source;
  
  factory HealthData.fromJson(Map<String, dynamic> json) {
    return HealthData(
      id: json['id'],
      patientId: json['patientId'],
      heartRate: json['heartRate'],
      spO2: json['spO2']?.toDouble(),
      steps: json['steps'],
      sleepMinutes: json['sleepMinutes'],
      timestamp: DateTime.parse(json['timestamp']),
      source: json['source'],
    );
  }
  
  Map<String, dynamic> toJson() => {
    'patientId': patientId,
    'heartRate': heartRate,
    'spO2': spO2,
    'steps': steps,
    'timestamp': timestamp.toIso8601String(),
  };
}
\end{lstlisting}

\section{Modèle User avec Rôles}
\begin{lstlisting}[style=code, caption=Modèle User Dart]
enum UserRole { ADMIN, DOCTOR, NURSE, PATIENT }

class User {
  final String id;
  final String email;
  final UserRole role;
  final String? firstName;
  final String? lastName;
  final String? phone;
  final bool enabled;
  
  bool get isAdmin => role == UserRole.ADMIN;
  bool get isDoctor => role == UserRole.DOCTOR;
  bool get isPatient => role == UserRole.PATIENT;
  
  factory User.fromJson(Map<String, dynamic> json) {
    return User(
      id: json['id'],
      email: json['email'],
      role: UserRole.values.firstWhere(
        (r) => r.name == json['role'],
        orElse: () => UserRole.PATIENT,
      ),
      firstName: json['firstName'],
      lastName: json['lastName'],
      enabled: json['enabled'] ?? true,
    );
  }
}
\end{lstlisting}

\chapter{Services et Communication API}

\section{Vue d'Ensemble des Services}
\begin{table}[H]
\centering
\caption{Services de l'application}
\begin{tabularx}{\textwidth}{|l|X|c|}
\hline
\textbf{Service} & \textbf{Responsabilité} & \textbf{Taille} \\
\hline
ApiService & Appels REST vers le backend (50+ endpoints) & 20.7 KB \\
\hline
AuthService & Authentification, login, register, logout & 7.3 KB \\
\hline
BleService & Communication Bluetooth avec SmartWatch & 3.7 KB \\
\hline
StorageService & Persistance locale (token, préférences) & 1.7 KB \\
\hline
MessageService & Gestion des messages et chat & 12.7 KB \\
\hline
NotificationService & Notifications push locales & 0.6 KB \\
\hline
ExportService & Export de données (PDF, CSV) & 0.7 KB \\
\hline
\end{tabularx}
\end{table}

\section{ApiService}
\begin{lstlisting}[style=code, caption=Extrait ApiService]
class ApiService {
  static const String baseUrl = 'http://10.0.2.2:8080/api';
  
  Future<Map<String, String>> _getHeaders() async {
    final token = await StorageService.getToken();
    return {
      'Content-Type': 'application/json',
      'Authorization': token != null ? 'Bearer $token' : '',
    };
  }
  
  Future<List<Patient>> getPatients() async {
    final response = await http.get(
      Uri.parse('$baseUrl/patients'),
      headers: await _getHeaders(),
    );
    
    if (response.statusCode == 200) {
      final List<dynamic> data = json.decode(response.body);
      return data.map((e) => Patient.fromJson(e)).toList();
    }
    throw Exception('Failed to load patients');
  }
  
  Future<void> submitHealthData(List<HealthData> dataList) async {
    final response = await http.post(
      Uri.parse('$baseUrl/smartwatch/health-data'),
      headers: await _getHeaders(),
      body: json.encode(dataList.map((e) => e.toJson()).toList()),
    );
    
    if (response.statusCode != 201) {
      throw Exception('Failed to submit health data');
    }
  }
}
\end{lstlisting}

\section{BleService - Connexion SmartWatch}
\begin{lstlisting}[style=code, caption=Service Bluetooth BLE]
class BleService {
  final Uuid _heartRateServiceUuid = 
      Uuid.parse("0000180d-0000-1000-8000-00805f9b34fb");
  final Uuid _heartRateCharacteristicUuid = 
      Uuid.parse("00002a37-0000-1000-8000-00805f9b34fb");
  
  Stream<List<DiscoveredDevice>> scanForDevices() {
    return _ble.scanForDevices(withServices: []);
  }
  
  Future<void> connectToDevice(String deviceId) async {
    _connectionStream = _ble.connectToDevice(id: deviceId);
  }
  
  Stream<int> subscribeToHeartRate(String deviceId) {
    final characteristic = QualifiedCharacteristic(
      characteristicId: _heartRateCharacteristicUuid,
      serviceId: _heartRateServiceUuid,
      deviceId: deviceId,
    );
    
    return _ble.subscribeToCharacteristic(characteristic)
        .map((data) => _parseHeartRate(data));
  }
  
  int _parseHeartRate(List<int> data) {
    if (data.isEmpty) return 0;
    final flags = data[0];
    final is16Bit = (flags & 0x01) != 0;
    if (is16Bit && data.length >= 3) {
      return data[1] | (data[2] << 8);
    } else if (data.length >= 2) {
      return data[1];
    }
    return 0;
  }
}
\end{lstlisting}

\chapter{Gestion d'État avec Provider}

\section{AuthProvider}
\begin{lstlisting}[style=code, caption=AuthProvider complet]
class AuthProvider extends ChangeNotifier {
  User? _currentUser;
  String? _token;
  bool _isLoading = false;
  
  User? get currentUser => _currentUser;
  bool get isAuthenticated => _token != null;
  bool get isLoading => _isLoading;
  
  Future<bool> login(String email, String password) async {
    _isLoading = true;
    notifyListeners();
    
    try {
      final response = await _authService.login(email, password);
      _token = response['token'];
      _currentUser = User.fromJson(response);
      
      await _storageService.saveToken(_token!);
      notifyListeners();
      return true;
    } catch (e) {
      return false;
    } finally {
      _isLoading = false;
      notifyListeners();
    }
  }
  
  Future<void> logout() async {
    _token = null;
    _currentUser = null;
    await _storageService.clearToken();
    notifyListeners();
  }
}
\end{lstlisting}

\chapter{Interface Utilisateur}

\section{Widgets Réutilisables}
\begin{table}[H]
\centering
\caption{Widgets personnalisés}
\begin{tabularx}{\textwidth}{|l|X|}
\hline
\textbf{Widget} & \textbf{Usage} \\
\hline
PatientCard & Carte affichant les infos patient avec avatar \\
\hline
HealthMetricCard & Carte métrique avec icône et valeur \\
\hline
AlertBadge & Badge coloré selon sévérité de l'alerte \\
\hline
VitalSignChart & Graphique fl\_chart pour données vitales \\
\hline
LoadingIndicator & Indicateur de chargement personnalisé \\
\hline
CustomTextField & Champ de saisie avec validation \\
\hline
GradientButton & Bouton avec dégradé de couleurs \\
\hline
StatCard & Carte statistique avec tendance \\
\hline
\end{tabularx}
\end{table}

\section{Thème et Couleurs}
\begin{lstlisting}[style=code, caption=Configuration du thème]
class AppTheme {
  static ThemeData lightTheme = ThemeData(
    primaryColor: Color(0xFF2962FF),
    colorScheme: ColorScheme.light(
      primary: Color(0xFF2962FF),
      secondary: Color(0xFF00C853),
      error: Color(0xFFFF5252),
    ),
    scaffoldBackgroundColor: Color(0xFFF8F9FA),
    cardTheme: CardTheme(
      elevation: 4,
      shape: RoundedRectangleBorder(
        borderRadius: BorderRadius.circular(16),
      ),
    ),
    appBarTheme: AppBarTheme(
      elevation: 0,
      backgroundColor: Colors.transparent,
      foregroundColor: Color(0xFF2D3436),
    ),
  );
}
\end{lstlisting}

\section{Localisation Multilingue}
L'application supporte 3 langues avec support RTL pour l'arabe :
\begin{table}[H]
\centering
\caption{Langues supportées}
\begin{tabular}{|l|l|l|}
\hline
\textbf{Langue} & \textbf{Code} & \textbf{Direction} \\
\hline
Français & fr & LTR \\
\hline
Anglais & en & LTR \\
\hline
Arabe & ar & RTL \\
\hline
\end{tabular}
\end{table}

\chapter{Diagrammes UML Frontend}

\section{Diagramme de Classes - Modèles}
\begin{figure}[H]
\centering
\resizebox{\textwidth}{!}{
\begin{tikzpicture}[
classbox/.style={rectangle, draw=#1, line width=2pt, fill=white, minimum width=4.5cm, inner sep=0pt, rounded corners=3pt},
assoc/.style={->, line width=2pt, >=stealth, #1}]

\node[classbox=clinalertblue] (user) at (0,8) {
\begin{tabular}{c}
\cellcolor{clinalertblue!25} \textbf{User} \\
\hline
\texttt{+ id: String} \\
\texttt{+ email: String} \\
\texttt{+ role: UserRole} \\
\texttt{+ firstName: String?} \\
\hline
\texttt{+ fromJson()} \\
\texttt{+ toJson()} \\
\end{tabular}};

\node[classbox=clinalertgreen] (patient) at (-5,4) {
\begin{tabular}{c}
\cellcolor{clinalertgreen!25} \textbf{Patient} \\
\hline
\texttt{+ id: String} \\
\texttt{+ name: String} \\
\texttt{+ age: int} \\
\texttt{+ doctorId: String} \\
\hline
\texttt{+ fromJson()} \\
\end{tabular}};

\node[classbox=clinred] (health) at (0,4) {
\begin{tabular}{c}
\cellcolor{clinred!20} \textbf{HealthData} \\
\hline
\texttt{+ heartRate: int?} \\
\texttt{+ spO2: double?} \\
\texttt{+ steps: int?} \\
\texttt{+ timestamp: DateTime} \\
\hline
\texttt{+ fromJson()} \\
\texttt{+ toJson()} \\
\end{tabular}};

\node[classbox=clinorange] (alert) at (5,4) {
\begin{tabular}{c}
\cellcolor{clinorange!20} \textbf{Alert} \\
\hline
\texttt{+ id: String} \\
\texttt{+ type: String} \\
\texttt{+ severity: String} \\
\texttt{+ message: String} \\
\hline
\texttt{+ fromJson()} \\
\end{tabular}};

\draw[assoc=clinalertgreen] (patient.north) -- node[left] {*} (user.south west);
\draw[assoc=clinred] (patient.east) -- node[above] {1..*} (health.west);
\draw[assoc=clinorange] (patient.south east) -- node[above] {0..*} (alert.south west);
\end{tikzpicture}}
\caption{Diagramme de Classes - Modèles Flutter}
\end{figure}

\section{Diagramme de Séquence - Login}
\begin{figure}[H]
\centering
\begin{tikzpicture}[scale=0.9]
\node[draw, rectangle, fill=clinalertblue!20, minimum width=2cm] (ui) at (0,0) {LoginScreen};
\node[draw, rectangle, fill=clinalertgreen!20, minimum width=2cm] (prov) at (4,0) {AuthProvider};
\node[draw, rectangle, fill=clinorange!20, minimum width=2cm] (api) at (8,0) {ApiService};
\node[draw, rectangle, fill=clingray!30, minimum width=2cm] (backend) at (12,0) {Backend};

\draw[dashed] (ui) -- (0,-9);
\draw[dashed] (prov) -- (4,-9);
\draw[dashed] (api) -- (8,-9);
\draw[dashed] (backend) -- (12,-9);

\draw[->, thick, clinalertblue] (0,-1) -- node[above, font=\small] {login(email, pwd)} (4,-1);
\draw[->, thick, clinalertgreen] (4,-2) -- node[above, font=\small] {authService.login()} (8,-2);
\draw[->, thick, clinorange] (8,-3) -- node[above, font=\small] {POST /api/auth/login} (12,-3);
\draw[<--, thick, clinorange] (8,-4) -- node[above, font=\small] {\{token, user\}} (12,-4);
\draw[->, thick, clingray] (4,-5) -- node[above, font=\scriptsize] {storage.saveToken()} (4,-5.5);
\draw[->, thick, clingray] (4,-6) -- node[above, font=\scriptsize] {notifyListeners()} (4,-6.5);
\draw[<--, thick, clinalertgreen] (0,-7) -- node[above, font=\small] {success} (4,-7);
\draw[->, thick, clinalertblue] (0,-8) -- node[above, font=\scriptsize] {Navigator.push(Dashboard)} (0,-8.5);
\end{tikzpicture}
\caption{Diagramme de Séquence - Processus de connexion}
\end{figure}

% ANNEXE - CAPTURES D'ÉCRAN
\chapter{Annexe : Captures d'Écran de l'Application}

Cette annexe présente les captures d'écran réelles de l'application ClinAlert Mobile, organisées par fonctionnalité.

\section{Écrans d'Authentification}

\begin{figure}[H]
\centering
\includegraphics[width=0.4\textwidth]{images/ecran_accueil.jpg}
\caption{Écran d'Accueil - Page de bienvenue avec options de connexion}
\end{figure}

\begin{figure}[H]
\centering
\includegraphics[width=0.4\textwidth]{images/ecran_connexion.jpg}
\caption{Écran de Connexion - Formulaire d'authentification}
\end{figure}

\begin{figure}[H]
\centering
\includegraphics[width=0.4\textwidth]{images/ecran_inscription.jpg}
\caption{Écran d'Inscription - Création de compte utilisateur}
\end{figure}

\section{Tableaux de Bord}

\begin{figure}[H]
\centering
\includegraphics[width=0.4\textwidth]{images/tableau_bord_medecin.jpg}
\caption{Tableau de Bord Médecin - Vue principale avec statistiques}
\end{figure}

\begin{figure}[H]
\centering
\includegraphics[width=0.4\textwidth]{images/tableau_bord_patient.jpg}
\caption{Tableau de Bord Patient - Données personnelles de santé}
\end{figure}

\section{Gestion des Patients}

\begin{figure}[H]
\centering
\includegraphics[width=0.4\textwidth]{images/liste_patients.jpg}
\caption{Liste des Patients - Affichage avec recherche}
\end{figure}

\begin{figure}[H]
\centering
\includegraphics[width=0.4\textwidth]{images/detail_patient.jpg}
\caption{Détail Patient - Informations complètes}
\end{figure}

\begin{figure}[H]
\centering
\includegraphics[width=0.4\textwidth]{images/ajout_patient.jpg}
\caption{Ajout/Modification Patient - Formulaire de saisie}
\end{figure}

\section{Données de Santé}

\begin{figure}[H]
\centering
\includegraphics[width=0.4\textwidth]{images/donnees_sante.jpg}
\caption{Données de Santé - Affichage des métriques vitales}
\end{figure}

\begin{figure}[H]
\centering
\includegraphics[width=0.4\textwidth]{images/graphique_cardiaque.jpg}
\caption{Graphique Cardiaque - Évolution du rythme cardiaque}
\end{figure}

\begin{figure}[H]
\centering
\includegraphics[width=0.4\textwidth]{images/historique_mesures.jpg}
\caption{Historique des Mesures - Liste chronologique}
\end{figure}

\begin{figure}[H]
\centering
\includegraphics[width=0.4\textwidth]{images/statistiques_sante.jpg}
\caption{Statistiques de Santé - Résumé des données}
\end{figure}

\begin{figure}[H]
\centering
\includegraphics[width=0.4\textwidth]{images/resume_quotidien.jpg}
\caption{Résumé Quotidien - Bilan journalier de santé}
\end{figure}

\begin{figure}[H]
\centering
\includegraphics[width=0.4\textwidth]{images/saisie_manuelle.jpg}
\caption{Saisie Manuelle - Enregistrement des signes vitaux}
\end{figure}

\section{SmartWatch et Bluetooth}

\begin{figure}[H]
\centering
\includegraphics[width=0.4\textwidth]{images/connexion_smartwatch.jpg}
\caption{Connexion SmartWatch - Interface d'appairage}
\end{figure}

\begin{figure}[H]
\centering
\includegraphics[width=0.4\textwidth]{images/scan_bluetooth.jpg}
\caption{Scan Bluetooth - Détection des appareils BLE}
\end{figure}

\section{Alertes et Notifications}

\begin{figure}[H]
\centering
\includegraphics[width=0.4\textwidth]{images/alertes_medicales.jpg}
\caption{Alertes Médicales - Liste des notifications d'anomalies}
\end{figure}

\section{Paramètres et Administration}

\begin{figure}[H]
\centering
\includegraphics[width=0.4\textwidth]{images/parametres.jpg}
\caption{Paramètres - Configuration de l'application}
\end{figure}

\begin{figure}[H]
\centering
\includegraphics[width=0.4\textwidth]{images/profil_utilisateur.jpg}
\caption{Profil Utilisateur - Informations personnelles}
\end{figure}

\begin{figure}[H]
\centering
\includegraphics[width=0.4\textwidth]{images/choix_langue.jpg}
\caption{Choix de la Langue - Support multilingue (FR/EN/AR)}
\end{figure}

\begin{figure}[H]
\centering
\includegraphics[width=0.4\textwidth]{images/gestion_utilisateurs.jpg}
\caption{Gestion des Utilisateurs - Administration des comptes (Admin)}
\end{figure}

\begin{figure}[H]
\centering
\includegraphics[width=0.4\textwidth]{images/gestion_cliniques.jpg}
\caption{Gestion des Cliniques - Administration des établissements}
\end{figure}

% CONCLUSION
\chapter*{Conclusion Générale}
\addcontentsline{toc}{chapter}{Conclusion Générale}

\section*{Récapitulatif}
Le système ClinAlert offre une solution complète et moderne pour le suivi médical :

\begin{itemize}
\item \ding{51} \textbf{Backend robuste} : Spring Boot 3.2, 9 entités JPA, 50+ endpoints REST
\item \ding{51} \textbf{Frontend moderne} : Flutter 3.x, 27 écrans, Provider, multilingue
\item \ding{51} \textbf{Sécurité} : Authentification JWT, gestion des rôles
\item \ding{51} \textbf{IoT} : Intégration Bluetooth BLE complète
\item \ding{51} \textbf{Données} : 12 métriques de santé, alertes automatiques
\end{itemize}

\section*{Améliorations Futures}
\begin{successbox}[Évolutions prévues]
\begin{enumerate}
\item \textbf{Tests} : Couverture > 80\% (unitaires et intégration)
\item \textbf{Notifications push} : Firebase Cloud Messaging
\item \textbf{Mode hors-ligne} : Cache local avec synchronisation
\item \textbf{WebSocket} : Alertes temps réel
\item \textbf{Monitoring} : Prometheus/Grafana
\item \textbf{CI/CD} : Pipeline automatisé GitHub Actions
\end{enumerate}
\end{successbox}

\vspace{1cm}
\begin{center}
{\color{clinalertblue}\rule{12cm}{2pt}}\\[0.8cm]
{\fontsize{30}{35}\selectfont\bfseries\textcolor{clinalertblue}{ClinAlert}}\\[0.5cm]
{\Large\textcolor{clingray}{Healthcare Monitoring System}}\\[0.3cm]
{\textcolor{clinalertgreen}{Suivi Médical Intelligent}}
\end{center}

\end{document}
